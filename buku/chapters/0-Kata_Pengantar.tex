\chapter{Kata Pengantar}
		Bismillahirrahmanirrahim, \\
		Alhamdulillahirabbil’alamin, segala puji bagi Allah SWT yang telah melimpahkan rahmat dan hidayah-Nya sehingga penulis dapat menyelesaikan Tugas Akhir yang berjudul "\textbf{Implementasi \textit{Headless Browser} untuk \textit{Load Testing} Berbasis \textit{Web Service} Menggunakan Infrastruktur \textit{Docker}}". Pengerjaan Tugas Akhir ini merupakan kesempatan yang baik bagi penulis untuk belajar lebih banyak, serta memperdalam apa yang telah didapatkan penulis selama menempuh perkuliahan di Informatika ITS. Dengan adanya Tugas Akhir ini, penulis juga dapat menghasilkan suatu implementasi dari apa yang penulis pelajari. Sehingga penulis mengharapkan implementasi Tugas Akhir ini dapat memberikan kontribusi dan manfaat. Selesainya Tugas Akhir ini tidak lepas dari bantuan dan dukungan dari beberapa pihak baik langsung maupun tidak langsung. Sehingga pada kesempatan ini penulis mengucapkan terima kasih kepada:
		\begin{enumerate}
			\item Allah SWT atas anugerah-Nya yang tidak terkira dan Nabi Muhammad SAW.
			\item Ibu dan Ayah yang selalu memberikan doa yang tak terhingga, dukungan moral dan material selama penulis hidup. Serta selalu memberi semangat dan dorongan untuk menyelesaikan Tugas Akhir ini.
			\item Bapak Royyana Muslim Ijtihadie, S.Kom, M.Kom, Ph.D., selaku pembimbing I dan Bapak Bagus Jati Santoso, S.Kom., Ph.D., selaku dosen pembimbing II yang telah membantu, membimbing dan memotivasi penulis mulai dari pengerjaan proposal hingga terselesaikannya Tugas Akhir ini.
			\item Bapak Darlis Herumurti, S.Kom., M.Kom., selaku Kepala Jurusan Teknik Informatika ITS saat ini, Bapak Radityo Anggoro, S.Kom., M.Sc., selaku koordinator TA dan segenap dosen dan karyawan Teknik Informatika yang telah memberikan ilmu dan pengalamannya, serta memberikan fasilitas yang aman dan nyaman, sehingga penulis dapat menempuh studi di Informatika ITS.
			\item Seluruh teman-teman Laboratorium Arsitektur dan Jaringan Komputer (AJK): Putol, Sinyo, Penyok, Satria, Nahda, Hana, Daus, Aguel, Raldo, Tamtam, Khawari, Haura, Lia, Sulton, Fawwaz, Yoga dan Mail yang bersedia direpotkan, merepotkan dan menemani penulis dalam mengerjakan Tugas Akhir.
			\item Alumni-alumni AJK, Mas Daniel, Mas Uul, Mas Wicak, Asbun, Toni, Pak Lek, Mas Fatih, Mbak Nindy dan Mbak Zaza yang selalu membuat saya termotivasi untuk lulus dan memberikan arahan dalam pengerjaan Tugas Akhir ini.
			\item Rozana Firdausi yang selalu merepotkan, direpotkan, menemani dan memberikan semangat secara langsung dan tidak langsung, sehingga penulis dapat mengerjakan Tugas Akhir ini dengan baik.
			\item Tim Proyek Arya Fajar Production, Pak Arya Yudhi Wijaya, S.Kom, M.Kom., Pak Fajar Baskoro, S.Kom., MT., Pak Ary Mazharuddin S., S.Kom., M.Comp.Sc., Mas Jumali, Tayar, Sinyo, Fajri, Adit, Irsa, Tamtam, Fasma, Jonathan, Nila, Andhika dan Fawwaz yang telah merepotkan dan memberikan pengalaman yang berharga bagi penulis dalam hal lapangan pekerjaan secara nyata.
			\item Grup Sayang: Sinyo, Yoza, Tayar, Fajri, Yayan, Alvin, Bas, Putol, Chasni, Raca dan reza yang telah memberikan secuil motivasi dan memberikan dampak yang baik secara jasmani maupun secara rohani penulis.
			\item Teman-teman satu bimbingan, Hero, Didin, Satria dan Nahda yang telah sama-sama berjuang dalam mengerjakan dan mengimplementasikan Tugas Akhir tentang Sistem Terdistribusi dan Komputasi Awan.
			\item Teman-teman angkatan 2015 yang menemani dari awal hingga akhir serta semua pihak yang turut membantu penulis dalam menyelesaikan Tugas Akhir ini.
		\end{enumerate}
		
		
		\indent Penulis sangat menyadari bahwa Tugas Akhir ini masih memiliki banyak kekurangan. Sehingga penulis memohon maaf secara tulus kepada pembaca. Dengan kerendahan hati, penulis mengharapkan kritik dan saran yang membangun dari pembaca sebagai pembelajaran dan perbaikan untuk kedepannya. \\

		\hfill Surabaya, Juni 2019 \\ \\
		
		\hfill Cahya Putra Hikmawan