\chapter{PENUTUP}
    Bab ini membahas kesimpulan yang dapat diambil dari tujuan pembuatan sistem dan hubungannya dengan hasil uji coba dan evaluasi yang telah dilakukan. Selain itu, terdapat beberapa saran yang bisa dijadikan acuan untuk melakukan pengembangan dan penelitian lebih lanjut.
        
	\section{Kesimpulan}
        Dari proses perencangan, implementasi dan pengujian terhadap sistem, dapat diambil beberapa kesimpulan berikut:
        \begin{enumerate}
        	\item Sistem ini dapat melakukan uji beban web menggunakan \textit{Headless Chrome} sebagai \textit{tester}, yang dipadukan dengan \textit{API} pengambil uji beban yaitu \textit{Puppeteer}.
        	
        	\item Sistem dapat menggunakan \textit{Docker} sebagai \textit{load generator} untuk pengganti \textit{resource user} dengan adanya \textit{Docker Image} yang sudah terpasang \textit{Node} dan \textit{Puppeteer} didalamnya.
        	
        	\item Sistem mengambil data atau laporan uji beban menggunakan \textit{Puppeteer} berupa \textit{performance metrics} dalam satuan \textit{millisecond}, \textit{error console} dan tangkapan layar web yang diuji.
        	
        	\item Dalam menangani layanan uji beban yang dikirimkan oleh pengembang web, sistem menggunakan sebuah \textit{task scheduler crontab} yang menjalankan \textit{script Python} setiap satu menit dan melakukan pengecekan secara berkala.
        	
        	\item Sistem pengujian beban aplikasi web yang dibangun menggunakan \textit{Docker} sebagai \textit{load generator} membutuhkan RAM sebesar 1,3MB setiap Docker Container, dan untuk CPU tidak terlalu berpengaruh setelah pembuatan karena Docker Container berada dalam kondisi sleep.
        	
        	\item Sistem ketika menerima request uji beban membutuhkan sekitar 30-55\% penggunaan CPU dan 1-1,4GB penggunaan RAM dalam melakukan uji beban dengan jumlah 100,200,300,400 dan 500. Tidak terlalu banyak sumberdaya yang dipakai, karena adanya pembatasan jalannya proses.
        \end{enumerate}
        
	\section{Saran}
		Berikut beberapa saran yang diberikan untuk pengembangan lebih lanjut:
		\begin{itemize}
			\item Melakukan optimasi terhadap penggunaan \textit{CPU} dan \textit{RAM} supaya hasil uji lebih baik, misalnya menggunakan \textit{multithreading} untuk mengatur proses ketika dijalankan.
			\item Memadukan \textit{Headless Chrome} dan \textit{Puppeteer} dengan \textit{tools load test} yang lain misalnya \textit{K6}, \textit{Jmeter Rest API}, \textit{Selenium Web Driver} atau \textit{Firefox Headless Mode} untuk mendapatkan data uji beban.
			\item Untuk menangani permintaan jumlah \textit{load generator} yang tinggi, server yang digunakan harus dilakukan \textit{scaling} secara horizontal maupun vertikal, supaya dapat menunjang sistem untuk melayani uji beban.
			\item Pengembang bisa mempercantik tampilan antarmuka pengguna agar lebih nyaman untuk digunakan dan mengoptimasi layanan uji beban pada metode \textit{POST}.
		\end{itemize}