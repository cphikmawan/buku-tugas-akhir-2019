\chapter{PENUTUP}
    Bab ini membahas kesimpulan yang dapat diambil dari tujuan pembuatan sistem dan hubungannya dengan hasil uji coba dan evaluasi yang telah dilakukan. Selain itu, terdapat beberapa saran yang bisa dijadikan acuan untuk melakukan pengembangan dan penelitian lebih lanjut.
        
	\section{Kesimpulan}
        Dari proses perencangan, implementasi dan pengujian terhadap sistem, dapat diambil beberapa kesimpulan berikut:
        \begin{enumerate}
        	\item Sistem dapat membuat \textit{tester} menggunakan \textit{Headless Chrome}, dengan cara memadukan \textit{Headless Chrome} dengan sebuah \textit{API} yang bisa menjadi pengambil uji beban pada \textit{browser}, yaitu \textit{Puppeteer}.
        	\item Sistem dapat menggunakan \textit{Docker} sebagai \textit{load generator}, dengan cara membuat sebuah \textit{Docker Image} yang sudah terpasang \textit{Node.js}, \textit{Chrome} dan \textit{Puppeteer} di dalamnya.
        	\item Sistem menyajikan laporan uji beban pada \textit{web service} dalam bentuk tabel. Laporan yang disajikan juga menampilkan \textit{error console} yang terekam pada \textit{browser}, serta tangkapan layar web yang diuji.
        	\item Dalam menangani layanan pengujian \textit{multiuser}, digunakan sebuah \textit{task scheduler crontab} yang menjalankan \textit{script Python} setiap satu menit dan melakukan pengecekan secara berkala.
        \end{enumerate}
        
	\section{Saran}
		Berikut beberapa saran yang diberikan untuk pengembangan lebih lanjut:
		\begin{itemize}
			\item Sistem sudah bisa membuat sebuah \textit{tester} dan \textit{load generator}, namun masih memiliki kendala terhadap penggunaan \textit{CPU} dan \textit{RAM} ketika menjalankan \textit{Headless Chrome} secara bersama, oleh karena itu pengembang bisa menambahkan sistem untuk melakukan \textit{multithreading} ketika \textit{load generator} dijalankan bersama, untuk saat ini sistem hanya membatasi penggunaan \textit{load generator}.
			\item \textit{Headless Chrome} dan \textit{Puppeteer} tergolong tools yang baru, namun perkembangannya sangat cepat. Untuk kedepannya pengembang bisa memadukan \textit{Headless Chrome} dan \textit{Puppeteer} dengan \textit{tools load test} yang lain misalnya \textit{K6}, \textit{Jmeter Rest API}, \textit{Selenium} atau \textit{Firefox Headless Mode}.
			\item Untuk menangani permintaan jumlah \textit{load generator} yang tinggi, server yang digunakan harus diperbanyak agar dapat menunjang sistem untuk melayani uji beban.
			\item Sistem dapat melakukan uji beban dengan metode \textit{GET}, oleh karena itu bisa dikembangkan untuk melayani uji beban menggunakan metode yang lain misalnya \textit{POST}. Serta pengembang juga bisa mempercantik tampilan antarmuka pengguna agar lebih nyaman untuk digunakan.
		\end{itemize}