\chapter{PENUTUP}
    Bab ini membahas kesimpulan yang dapat diambil dari tujuan pembuatan sistem dan hubungannya dengan hasil uji coba dan evaluasi yang telah dilakukan. Selain itu, terdapat beberapa saran yang bisa dijadikan acuan untuk melakukan pengembangan dan penelitian lebih lanjut.
        
	\section{Kesimpulan}
        Dari proses perencangan, implementasi dan pengujian terhadap sistem, dapat diambil beberapa kesimpulan berikut:
        \begin{enumerate}
        	\item Dalam pengembangan web, uji beban dapat dilakukan menggunakan \textit{Headless Chrome} sebagai \textit{tester}, yang dipadukan dengan \textit{API} pengambil uji beban yaitu \textit{Puppeteer}.
     		\item Load generator yang digunakan sebagai resource user dapat diimplementasikan oleh Docker. Docker dipasang menggunakan Docker Image yang sudah diinstalasi \textit{Node.js} dan \textit{Puppeteer} didalamnya.
        	\item Data Laporan uji beban didapatkan menggunakan \textit{Puppeteer} dan disajikan pada web dalam bentuk tabel dengan data inti yaitu \textit{Response End}, \textit{CSS Tracing End}, \textit{Dom Content Loaded}, \textit{First Meaningful Paint} dan \textit{Load Event End}. Laporan yang disajikan juga menampilkan \textit{error console} yang terekam pada \textit{browser} dan tangkapan layar web yang diuji.
        	\item Dalam menangani layanan uji beban yang dikirimkan oleh pengembang web, sistem menggunakan sebuah \textit{task scheduler crontab} yang menjalankan \textit{script Python} setiap satu menit dan melakukan pengecekan secara berkala.
        \end{enumerate}
        
	\section{Saran}
		Berikut beberapa saran yang diberikan untuk pengembangan lebih lanjut:
		\begin{itemize}
			\item Melakukan optimasi terhadap penggunaan \textit{CPU} dan \textit{RAM} supaya hasil uji lebih baik, misalnya menggunakan \textit{multithreading} untuk mengatur proses ketika dijalankan.
			\item Memadukan \textit{Headless Chrome} dan \textit{Puppeteer} dengan \textit{tools load test} yang lain misalnya \textit{K6}, \textit{Jmeter Rest API}, \textit{Selenium Web Driver} atau \textit{Firefox Headless Mode} untuk mendapatkan data uji beban.
			\item Untuk menangani permintaan jumlah \textit{load generator} yang tinggi, server yang digunakan harus dilakukan \textit{scaling} secara horizontal maupun vertikal, supaya dapat menunjang sistem untuk melayani uji beban.
			\item Pengembang bisa mempercantik tampilan antarmuka pengguna agar lebih nyaman untuk digunakan dan mengoptimasi layanan uji beban pada metode \textit{POST}.
		\end{itemize}