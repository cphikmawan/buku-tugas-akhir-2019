\chapter{PENDAHULUAN}
	Pada bab ini akan dipaparkan mengenai garis besar Tugas Akhir yang meliputi latar belakang, tujuan, rumusan dan batasan permasalahan, metodologi pembuatan Tugas Akhir, dan sistematika penulisan.
        
	\section{Latar Belakang}
		Saat ini pengembangan aplikasi berbasis web sangat banyak dilakukan. Dalam pengembangan web, akan diperlukan uji beban yang dilakukan pada web yang dikembangkan sebelum diluncurkan ke tahap produksi. Salah satu teknik uji beban adalah \textit{Headless Testing}, salah satu \textit{browser} yang menggunakan teknik ini adalah \textit{Headless Chrome}. Teknik ini menyediakan akses kontrol seperti \textit{browser} pada umumnya untuk mendapatkan data dokumen uji beban, hanya saja teknik ini berjalan secara \textit{headless} atau tanpa menampilkan antarmuka pengguna. Dikarenakan berjalan secara \textit{headless}, maka untuk melakukan pengambilan data dokumen uji beban dilakukan menggunakan baris perintah atau \textit{CLI(Command Line Interface)}.
		
		\indent Pengembang aplikasi web tentu saja membutuhkan teknik untuk uji beban, namun dalam melakukan uji beban dibutuhkan \textit{resource user} untuk mengakses web dan waktu yang cukup lama karena dilakukan secara manual. Oleh karena itu, dibutuhkan suatu sistem yang dapat melakukan automasi untuk uji beban web, membuat \textit{load generator} yang digunakan sebagai \textit{resource user} dan dapat mempersingkat waktu uji beban.
		
		\indent Pada tugas akhir ini, akan dibangun sebuah sistem uji beban menggunakan teknik secara \textit{headless} yang akan memanfaatkan \textit{Headless Chrome} sebagai \textit{tester} dan membuat \textit{load generator} yang memanfaatkan infrastruktur \textit{Docker} sebagai \textit{resource user}, serta automasi pengambilan data dari \textit{tester} menggunakan pustaka \textit{Node} yaitu \textit{Puppeteer}\cite{puppeteer}.

	\section{Rumusan Masalah}
       	Rumusan masalah yang diangkat dalam tugas akhir ini dapat dipaparkan sebagai berikut :
		\begin{enumerate}
			\item Bagaimana cara mengimplementasikan \textit{Headless Chrome} sebagai \textit{tester} untuk \textit{load generator}?
			\item Bagaimana cara mengimplementasikan \textit{Docker} bisa menjadi \textit{load generator} untuk uji beban?
			\item Bagaimana cara menghasilkan laporan uji beban pada web?
			\item Bagaimana mengelola layanan pengujian untuk \textit{multiuser} dalam bentuk antrian?
		\end{enumerate}

	\section{Batasan Masalah}
		Dari permasalahan yang telah dipaparkan di atas, terdapat beberapa batasan masalah pada tugas akhir ini, yaitu:
		\begin{enumerate}
			\item \textit{Headless Browser} yang digunakan adalah \textit{Headless Chrome}.
			\item Kontainer yang digunakan adalah \textit{Docker}.
			\item Distribusi kontainer menggunakan \textit{Docker Swarm}.
			\item Aplikasi yang akan diuji berupa aplikasi web.
			\item Uji coba aplikasi akan menggunakan  \textit{Node library yang menyediakan API (Application Programming Interface)} untuk mengontrol \textit{Headless Chrome} yaitu \textit{Puppeteer}.
		\end{enumerate}

	\section{Tujuan}
       	Tujuan pembuatan tugas akhir ini antara lain:
       	\begin{enumerate}
       		\item Membuat sistem manajemen pengujian aplikasi secara \textit{headless} menggunakan \textit{Headless Chrome}.
       		\item Membuat sistem agar \textit{Docker} bisa menjadi \textit{load generator} untuk pengujian.
       		\item Mengimplementasikan pengujian menggunakan skenario yang sudah disiapkan.
       		\item Membuat sistem untuk menampilkan laporan uji beban pada web.
       	\end{enumerate}
        
	\section{Manfaat}
		Manfaat dari pembuatan tugas akhir ini yaitu:
		\begin{enumerate}
			\item Mempelajari penggunaan \textit{Headless Browser} untuk pengujian suatu aplikasi yaitu \textit{Headless Chrome}
			\item Meminimalisir adanya kegagalan web saat dimuat ketika sudah diluncurkan.
			\item Mengetahui performa \textit{load} suatu aplikasi web.
		\end{enumerate}
	
	\section{Metodologi}
		Metodologi yang digunakan untuk pembuatan Tugas Akhir ini adalah sebagai berikut:	
		\subsection{Penyusunan Proposal Tugas Akhir}
			Proposal tugas akhir ini berisi tentang deskripsi pendahuluan dari tugas
			akhir yang akan dibuat. Pendahuluan ini terdiri dari hal yang menjadi latar
			belakang diajukannya usulan tugas akhir, rumusan masalah yang diangkat,
			batasan masalah untuk tugas akhir, tujuan dari pembuatan tugas akhir, dan
			manfaat dari hasil pembuatan tugas akhir. Selain itu dijabarkan pula tinjauan
			pustaka yang digunakan sebagai referensi pendukung pembuatan tugas akhir.
			Sub bab metodologi berisi penjelasan mengenai tahapan penyusunan tugas
			akhir mulai dari penyusunan proposal hingga penyusunan buku tugas akhir.
			Terdapat pula sub bab jadwal kegiatan yang menjelaskan jadwal pengerjaan
			tugas akhir.	
		\subsection{Studi Literatur}
			Pada tahap ini dilakukan pencarian informasi dan referensi mengenai \textit{Headless Chrome}, \textit{Puppeteer} dan \textit{Docker} untuk mendukung dan memastikan setiap tahap pembuatan tugas akhir sesuai dengan standar dan konsep yang berlaku, serta dapat diimplementasikan. Sumber informasi dan referensi bisa didapatkan dari buku, jurnal dan internet.
		\subsection{Analisis dan Desain Perangkat Lunak}
			Pada tahap ini dilakukan analisis dan perancangan terhadap arsitektur sistem yang akan dibuat. Tahap ini merupakan tahap yang paling penting dimana segala bentuk implementasi bisa bekerja dengan baik ketika arsitektur sistem yang baik pula.
		\subsection{Implementasi Perangkat Lunak}
			Pada tahap ini dilakukan implementasi atau realisasi dari hasil analisis dan perancangan arsitektur yang sudah dibuat sebelumnya, sehingga menjadi sebuah infrastruktur yang sesuai dengan apa yang direncanakan. 
		\subsection{Pengujian dan Evaluasi}
			Pada tahap ini dilakukan pengujian untuk mengukur performa web dan kegagalan saat dimuat menggunakan arsitektur sistem yang sudah dibuat menggunakan infrastruktur \textit{Docker}. Beberapa performa yang diukur pada pengujian antara lain, \textit{load time}, \textit{response time}, \textit{firstmeaningfulpain}, \textit{css tracing}, \textit{domcontentloadevent} dan \textit{dominteractive} dalam satuan \textit{ms(millisecond)} serta \textit{error console}. Setelah dilakukan uji coba, maka dilakukan evaluasi terhadap kinerja arsitektur sistem yang telah diimplementasikan dengan harapan bisa diperbaiki ketika ada pengembangan ke depannya.
		\subsection{Penyusunan Buku Tugas Akhir}
			Pada tahap ini dilakukan penyusunan buku tugas akhir yang berisikan dokumentasi yang mencakup teori, konsep, implementasi dan hasil pengerjaan tugas akhir.
	
	\section{Sistematika Penulisan}
		Sistematika penulisan laporan tugas akhir secara garis besar adalah sebagai berikut:
		\begin{enumerate}
			\item Bab I. Pendahuluan\\
				Bab ini berisi penjelasan mengenai latar belakang, rumusan masalah, batasan masalah, tujuan, manfaat, metodologi dan sistematika penulisan dari pembuatan tugas akhir.
			\item Bab II. Tinjauan Pustaka\\
				Bab ini berisi kajian teori atau penjelasan metode, algoritme, \textit{library} dan \textit{tools} yang digunakan dalam penyusunan tugas akhir ini. Kajian teori yang dimaksud berisi tentang penjelasan singkat mengenai \textit{Headless Browser}, \textit{Headless Chrome}, \textit{Puppeteer}, \textit{NodeJS}, \textit{Docker}, \textit{Docker Swarm} dan \textit{Laravel}.
			\item Bab III. Desain dan Perancangan\\
				Bab ini berisi mengenai analisis dan perancangan arsitektur sistem yang akan diimplementasikan pada tugas akhir ini.
			\item Bab IV. Implementasi\\
				Bab ini berisi bahasan tentang implementasi dari arsitektur sistem yang dibuat pada bab sebelumnya. Penjelasan berupa kode program yang digunakan untuk mengimplementasikan sistem.
			\item Bab V. Pengujian dan Evaluasi\\
				Bab ini berisi bahasan tentang tahapan uji coba terhadap performa web dan evaluasi terhadap sistem yang dibuat.
			\item Bab VI. Penutup\\
				Bab ini merupakan bab terakhir yang memaparkan kesimpulan dari hasil pengujian dan evaluasi yang telah dilakukan. Pada bab ini juga terdapat saran yang ditujukan bagi pembaca yang berminat untuk melakukan pengembangan lebih lanjut.
			\item Daftar Pustaka\\
				Bab ini berisi daftar pustaka yang dijadikan literatur dalam tugas akhir.
			\item Lampiran\\
				Dalam lampiran terdapat kode sumber program secara keseluruhan.
		\end{enumerate}