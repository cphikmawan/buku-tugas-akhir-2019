\chapter{INSTALASI PERANGKAT LUNAK}

\section*{Instalasi \textit{Docker}}
	Untuk melakukan instalasi \textit{Docker}, dilakukan seperti langkah-langkah berikut:
\begin{lstlisting}[frame=single,tabsize=2,breaklines,caption={Perintah instalasi Docker },label=instalasidocker, captionpos=b, language=json,numbers=none]
$ sudo apt-get -y install \
apt-transport-https \
ca-certificates \
curl

$ curl -fsSL https://download.docker.com/linux/
ubuntu/gpg | sudo apt-key add -

$ sudo add-apt-repository \
"deb [arch=amd64] https://download.docker.com/linux/ubuntu \
$(lsb_release -cs) \
stable"

$ sudo apt-get update

$ sudo apt-get install docker-ce docker-ce-cli containerd.io
\end{lstlisting}
	
	Setelah menjalankan perintah pada kode sumber \ref{instalasidocker}. Jalankan perintah berikut agar \textit{Docker} bisa dijalankan sebagai \textit{Non-Root User}.
\begin{lstlisting}[frame=single,tabsize=2,breaklines,caption={Perintah mengubah hak User },label=nonrootuser, captionpos=b, language=json,numbers=none]
$ sudo groupadd docker
$ sudo usermod -aG docker $USER
\end{lstlisting}

\section*{Instalasi \textit{Docker Compose}}
\textit{Docker Compose} digunakan untuk automasi dalam menjalankan \textit{Dockerfile} dan berperan untuk pembuatan Docker Image. Instalasi \textit{Docker Compose} dapat dilihat pada kode sumber \ref{dockercomposeinstall}.

\begin{lstlisting}[frame=single,tabsize=2,breaklines,caption={Perintah instalasi Docker Compose},label=dockercomposeinstall, captionpos=b, language=json,numbers=none]
$ sudo curl -L "https://github.com/docker/compose/releases/download/1.24.0/docker-compose-$(uname -s)-$(uname -m)" -o /usr/local/bin/docker-compose

$ sudo chmod +x /usr/local/bin/docker-compose

$ sudo ln -s /usr/local/bin/docker-compose /usr/bin/docker-compose
\end{lstlisting}

\section*{Instalasi \textit{Linux Package}}
Beberapa \textit{package} yang dibutuhkan dalam pembuatan sistem.
\begin{itemize}
	\item Node.js \\
		\texttt{\$ sudo apt install nodejs}\\
	\item NPM \\
		\texttt{\$ sudo apt install npm}\\
	\item MySQL \\
		\texttt{\$ sudo apt install mysql-server mysql-client}\\
	\item PIP \\
		\texttt{\$ sudo apt install python3-pip}\\
	\item mysql-connector-python \\
		\texttt{\$ pip3 install mysql-connector-python}\\
	\item Composer \\
		\texttt{\$ php -r "copy('https://getcomposer.org/installer', 'composer-setup.php');"} \\ \\
		\texttt{\$ php -r "if (hash\_file('sha384', 'composer-setup.php') === '48e3236262b34d30969dca3c37281b3b4bbe3221bda826ac6a9a62d6444cdb0dcd0615698a5cbe587c3f0fe57a54d8f5') { echo 'Installer verified'; } else { echo 'Installer corrupt'; unlink('composer-setup.php'); } echo PHP\_EOL;"}\\ \\
		\texttt{\$ php composer-setup.php} \\
		\texttt{\$ php -r "unlink('composer-setup.php');"} \\
		\texttt{\$ mv composer.phar /usr/local/bin/composer} \\
	\item Laravel \\
		\texttt{\$ composer global require laravel/installer}\\
\end{itemize}