\chapter{INSTALASI PERANGKAT LUNAK}

\section*{Instalasi Docker}
	Untuk melakukan instalasi Docker, dilakukan seperti langkah-langkah berikut:
\begin{lstlisting}[frame=single,tabsize=2,breaklines,caption={Perintah untuk instalasi Docker },label=instalasidocker, captionpos=b, language=json,numbers=none]
$ sudo apt-get -y install \
apt-transport-https \
ca-certificates \
curl

$ curl -fsSL https://download.docker.com/linux/
ubuntu/gpg | sudo apt-key add -

$ sudo add-apt-repository \
"deb [arch=amd64] https://download.docker.com/linux/ubuntu \
$(lsb_release -cs) \
stable"

$ sudo apt-get update

$ sudo apt-get install docker-ce docker-ce-cli containerd.io
\end{lstlisting}
	
	Setelah menjalankan perintah pada kode sumber \ref{instalasidocker}. Jalankan perintah berikut agar Docker bisa dijalankan sebagai Non-Root User.
\begin{lstlisting}[frame=single,tabsize=2,breaklines,caption={Perintah untuk mengubah hak User },label=nonrootuser, captionpos=b, language=json,numbers=none]
$ sudo groupadd docker
$ sudo usermod -aG docker $USER
\end{lstlisting}

\section*{Instalasi Docker Compose}
Docker Compose digunakan untuk automasi dalam menjalankan Dockerfile dan berperan untuk pembuatan Docker Image. Instalasi Docker Compose dapat dilihat pada kode sumber \ref{dockercomposeinstall}.

\begin{lstlisting}[frame=single,tabsize=2,breaklines,caption={Perintah instalasi Docker Compose},label=dockercomposeinstall, captionpos=b, language=json,numbers=none]
$ sudo curl -L "https://github.com/docker/compose/releases/download/1.24.0/docker-compose-$(uname -s)-$(uname -m)" -o /usr/local/bin/docker-compose

$ sudo chmod +x /usr/local/bin/docker-compose

$ sudo ln -s /usr/local/bin/docker-compose /usr/bin/docker-compose
\end{lstlisting}
