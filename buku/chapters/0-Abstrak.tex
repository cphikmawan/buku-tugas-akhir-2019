\begin{abstrak}
		Saat ini pengembangan aplikasi berbasis web sangat banyak dilakukan. Dalam pengembangan web, akan diperlukan uji beban yang dilakukan pada web yang dikembangkan sebelum diluncurkan ke tahap produksi. Dilakukannya uji beban tentu saja untuk menghindari terjadinya kegagalan muat ketika sudah diluncurkan.
		
		\indent Dalam melakukan uji beban, pengembang akan membutuhkan resource user untuk mengakses web dan waktu yang cukup lama karena dilakukan secara manual. Beberapa contoh tools yang dapat mengatasi hal tersebut untuk melakukan uji beban yaitu JMeter, k6 dan sebagainya. Namun tools tersebut menggunakan thread untuk resource user-nya, bila dilihat kondisi lapangan yaitu pengguna mengakses web menggunakan browser dan memiliki IP masing-masing, thread kurang bisa mengatasi hal tersebut. Oleh karena itu, dibutuhkan tools yang dapat menangani hal tersebut, salah satunya yaitu pengujian secara headless yang memanfaatkan infrastruktur Docker dan melakukan otomasi untuk pengambilan data uji beban. Docker akan menjadi resource user yang mengakses melalui headless browser untuk uji beban.
		
		\indent Pada tugas akhir ini, dibuat sistem yang mampu melayani uji beban pada suatu web. Dengan menggunakan Docker Container sebagai load generator yang akan bisa melakukan akses ke web yang diuji melalui Headless Chrome dan akan dilakukan otomasi pengambilan data uji beban menggunakan API Puppeteer. Selain itu, sistem juga dilengkapi task scheduler untuk melayani permintaan uji beban dari multiuser. Hasil uji yang didapatkan pada sistem ini adalah beberapa performance metrics, error console pada browser dan tangkapan layar terhadap interface web yang diuji.
		
		\indent Hasil uji coba menunjukkan bahwa Docker Container dapat dijadikan sebagai resource user karena sumberdaya yang dibutuhkan hanya 1,3MB setiap Docker Container dalam keadaan sleep, namun untuk ketika Puppeteer mengakses web yang diuji melalui Headless Chrome membutuhkan sumberdaya CPU dan RAM yang cukup tinggi jika dijalankan secara bersama-sama. Maka dari itu, dibutuhkan pembatasan atau manajemen proses untuk Headless Chrome. Sedangkan untuk data uji beban yang dihasilkan dari Headless Chrome sangat memuaskan. \\

	\noindent \textbf{Kata-Kunci}: headless browser, headless chrome, load test, puppeteer, docker
\end{abstrak}

\newpage
\begin{abstract}
		Nowadays web-based application development is very much done. In web development, a load test will be needed on the web that was developed before being launched into the production stage. The load test is carried out of course to avoid loading failure when it is launched.
		
		\indent In carrying out load tests, developers will need a resource user to access the web and a considerable amount of time because it is done manually. Some examples of tools that can overcome this are to carry out load tests namely JMeter, k6 and so on. However, these tools use threads for their resource users, if you see field conditions, that is, users access the web using a browser and have their own IP, the thread is less able to overcome this. Therefore, we need tools that can handle this, one of which is headless testing that utilizes Docker infrastructure and automates the load test data collection. Docker will be a resource user who accesses through the headless browser for load testing.
		
		\indent In this final project, a system that is capable of serving load tests on a web is made. By using Docker Container as a load generator that will be able to access the web tested through Headless Chrome and automation of load test data retrieval will be carried out using the Puppeteer API. In addition, the system also features a task scheduler to serve multiuser load test requests. The test results obtained in this system are several performance metrics, browser console errors and screenshots of the tested web interface.
		
		\indent The trial results show that Docker Container can be used as a resource user because the resources needed are only 1.3MB per Docker Container in a sleep state, but for when Puppeteer accesses the web tested through Headless Chrome it requires high CPU and RAM resources if run together -same. Therefore, restrictions or process management are needed for Headless Chrome. Whereas for load test data generated from Headless Chrome is very satisfying. \\		

	\noindent \textbf{Keywords}: headless browser, headless chrome, load test, puppeteer, docker
\end{abstract}