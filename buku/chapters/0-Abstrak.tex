\begin{abstrak}
		Saat ini pengembangan aplikasi berbasis web sangat banyak dilakukan. Dalam pengembangan web, akan diperlukan uji beban yang dilakukan pada web yang dikembangkan sebelum diluncurkan ke tahap produksi. Dilakukannya uji beban tentu saja untuk menghindari terjadinya kegagalan muat ketika sudah diluncurkan.

		\indent Dalam melakukan uji beban, pengembang akan membutuhkan resource user untuk mengakses web dan waktu yang cukup lama karena dilakukan secara manual. Oleh karena itu, dibutuhkan suatu sistem yang dapat melakukan otomasi untuk uji beban web, membuat load generator yang digunakan sebagai resource user dan dapat mempersingkat waktu uji beban.
		
		\indent Pada tugas akhir ini, dibuat sistem yang mampu melayani uji beban pada suatu web. Dengan menggunakan Docker Container sebagai load generator yang akan bisa melakukan akses ke web yang diuji melalui Headless Chrome dan akan dilakukan otomasi pengambilan data uji beban menggunakan API Puppeteer. Selain itu sistem juga dilengkapi task scheduler untuk melayani permintaan uji beban dari multiuser. Hasil uji yang didapatkan pada sistem ini adalah beberapa performance metrics, error console pada browser dan tangkapan layar terhadap interface web yang diuji. \\

	\noindent \textbf{Kata-Kunci}: headless browser, headless chrome, load test, puppeteer, docker
\end{abstrak}
\newpage
\begin{abstract}
		Nowadays web-based application development is very much done. In web development, a load test will be needed on the web that was developed before being launched into the production stage. The load test is carried out of course to avoid loading failure when it is launched.
		
		\indent In carrying out load tests, developers will need a resource user to access the web and a considerable amount of time because it is done manually. Therefore, we need a system that can perform automation for web load testing, create load generators that are used as resource users and can shorten load test time.
		
		\indent In this final project, a system that is capable of serving load tests on a web is made. By using Docker Container as a load generator that will be able to access the web tested through Headless Chrome and automation of load test data retrieval will be done using the Puppeteer API. In addition, the system also has a task scheduler to serve multiuser load test requests. The test results obtained in this system are several performance metrics, browser console errors and screenshots of the tested web interface. \\

	\noindent \textbf{Keywords}: headless browser, headless chrome, load test, puppeteer, docker
\end{abstract}