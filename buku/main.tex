\documentclass[12pt,oneside,reqno]{ta-its}
\usepackage{hyperref}
\usepackage{listings}
\usepackage{float}
\usepackage{mdframed}
\usepackage{wrapfig}

\renewcommand{\lstlistingname}{Kode Sumber}
\renewcommand{\lstlistlistingname}{DAFTAR KODE SUMBER}

\title{Implementasi Headless Browser untuk Load Testing Berbasis Web Service Menggunakan Infrastruktur Docker
}{Headless Browser Implementation for Load Testing Based on Web Service Using Docker Infrastructure}{IF184802}

\author{Cahya Putra Hikmawan}{05111540000119}

\supervisorOne{Royyana Muslim Ijtihadie, S.Kom., M.Kom., PhD.}{197708242003041001}
\supervisorTwo{Bagus Jati Santoso, S.Kom., Ph.D}{198611252018031001}

\degree{Sarjana Komputer}{Komputasi Berbasis Jaringan}{S1}{Teknik Informatika}{Informatics}{Teknologi Informasi}{FTIf}{Information Technology}

\time{Juni}{2019}

\begin{document}
	\frontmatter % Halaman dengan penomoran romawi kecil
	\maketitle
	\legalityPaper % Lembar Pengesahan
	
    \begin{abstrak}
		Saat ini pengembangan aplikasi berbasis web sangat banyak dilakukan. Dalam pengembangan web, akan diperlukan uji beban yang dilakukan pada web yang dikembangkan sebelum diluncurkan ke tahap produksi. Dilakukannya uji beban tentu saja untuk menghindari terjadinya kegagalan muat ketika sudah diluncurkan.
		
		\indent Dalam melakukan uji beban, pengembang akan membutuhkan resource user untuk mengakses web dan waktu yang cukup lama karena dilakukan secara manual. Beberapa contoh tools yang dapat mengatasi hal tersebut untuk melakukan uji beban yaitu JMeter, k6 dan sebagainya. Namun tools tersebut menggunakan thread untuk resource user-nya, bila dilihat kondisi lapangan yaitu pengguna mengakses web menggunakan browser dan memiliki IP masing-masing, thread kurang bisa mengatasi hal tersebut. Oleh karena itu, dibutuhkan tools yang dapat menangani hal tersebut, salah satunya yaitu pengujian secara headless yang memanfaatkan infrastruktur Docker dan melakukan otomasi untuk pengambilan data uji beban. Docker akan menjadi resource user yang mengakses melalui headless browser untuk uji beban.
		
		\indent Pada tugas akhir ini, dibuat sistem yang mampu melayani uji beban pada suatu web. Dengan menggunakan Docker Container sebagai load generator yang akan bisa melakukan akses ke web yang diuji melalui Headless Chrome dan akan dilakukan otomasi pengambilan data uji beban menggunakan API Puppeteer. Selain itu, sistem juga dilengkapi task scheduler untuk melayani permintaan uji beban dari multiuser. Hasil uji yang didapatkan pada sistem ini adalah beberapa performance metrics, error console pada browser dan tangkapan layar terhadap interface web yang diuji.
		
		\indent Hasil uji coba menunjukkan bahwa Docker Container dapat dijadikan sebagai resource user karena sumberdaya yang dibutuhkan hanya 1,3MB setiap Docker Container dalam keadaan sleep, namun untuk ketika Puppeteer mengakses web yang diuji melalui Headless Chrome membutuhkan sumberdaya CPU dan RAM yang cukup tinggi jika dijalankan secara bersama-sama. Maka dari itu, dibutuhkan pembatasan atau manajemen proses untuk Headless Chrome. Sedangkan untuk data uji beban yang dihasilkan dari Headless Chrome sangat memuaskan. \\

	\noindent \textbf{Kata-Kunci}: headless browser, headless chrome, load test, puppeteer, docker
\end{abstrak}

\newpage
\begin{abstract}
		Nowadays web-based application development is very much done. In web development, a load test will be needed on the web that was developed before being launched into the production stage. The load test is carried out of course to avoid loading failure when it is launched.
		
		\indent In carrying out load tests, developers will need a resource user to access the web and a considerable amount of time because it is done manually. Some examples of tools that can overcome this are to carry out load tests namely JMeter, k6 and so on. However, these tools use threads for their resource users, if you see field conditions, that is, users access the web using a browser and have their own IP, the thread is less able to overcome this. Therefore, we need tools that can handle this, one of which is headless testing that utilizes Docker infrastructure and automates the load test data collection. Docker will be a resource user who accesses through the headless browser for load testing.
		
		\indent In this final project, a system that is capable of serving load tests on a web is made. By using Docker Container as a load generator that will be able to access the web tested through Headless Chrome and automation of load test data retrieval will be carried out using the Puppeteer API. In addition, the system also features a task scheduler to serve multiuser load test requests. The test results obtained in this system are several performance metrics, browser console errors and screenshots of the tested web interface.
		
		\indent The trial results show that Docker Container can be used as a resource user because the resources needed are only 1.3MB per Docker Container in a sleep state, but for when Puppeteer accesses the web tested through Headless Chrome it requires high CPU and RAM resources if run together -same. Therefore, restrictions or process management are needed for Headless Chrome. Whereas for load test data generated from Headless Chrome is very satisfying. \\		

	\noindent \textbf{Keywords}: headless browser, headless chrome, load test, puppeteer, docker
\end{abstract}
	\chapter{Kata Pengantar}
		Isi kata pengantar

		\hfill Surabaya, Juni 2019 \\ \\
		
		\hfill Cahya Putra Hikmawan

	\cleardoublepage % Mengisi penanda halaman genap yang kosong

	\tableofcontents % Daftar Isi
	\listoftables % Daftar Tabel
	\listoffigures % Daftar Gambar
	\lstlistoflistings % Daftar Kode Sumber

	\mainmatter
	\chapter{PENDAHULUAN}
	Pada bab ini akan dipaparkan mengenai garis besar tugas akhir yang meliputi latar belakang, tujuan, rumusan dan batasan permasalahan, metodologi pembuatan tugas akhir, dan sistematika penulisan.
        
	\section{Latar Belakang}
		Saat ini pengembangan aplikasi berbasis web sangat banyak dilakukan. Dalam pengembangan web, akan diperlukan uji beban yang dilakukan pada web yang dikembangkan sebelum diluncurkan ke tahap produksi. Salah satu teknik uji beban adalah \textit{Headless Testing}, salah satu \textit{browser} yang menggunakan teknik ini adalah \textit{Headless Chrome}. Teknik ini menyediakan akses kontrol seperti \textit{browser} pada umumnya untuk mendapatkan data dokumen uji beban, hanya saja teknik ini berjalan secara \textit{headless} atau tanpa menampilkan antarmuka pengguna. Dikarenakan berjalan secara \textit{headless}, maka untuk melakukan pengambilan data dokumen uji beban dilakukan menggunakan baris perintah atau \textit{CLI(Command Line Interface)}.
		
		\indent Pengembang aplikasi web tentu saja membutuhkan teknik untuk uji beban, namun dalam melakukan uji beban dibutuhkan \textit{resource user} untuk mengakses web dan waktu yang cukup lama karena dilakukan secara manual. Oleh karena itu, dibutuhkan suatu sistem yang dapat melakukan otomasi untuk uji beban web, membuat \textit{load generator} yang digunakan sebagai \textit{resource user} dan dapat mempersingkat waktu uji beban.
		
		\indent Pada tugas akhir ini, akan dibangun sebuah sistem uji beban menggunakan teknik secara \textit{headless} yang akan memanfaatkan \textit{Headless Chrome} sebagai \textit{tester} dan membuat \textit{load generator} yang memanfaatkan infrastruktur \textit{Docker} sebagai \textit{resource user}, serta otomasi pengambilan data dari \textit{tester} menggunakan pustaka \textit{Node} yaitu \textit{Puppeteer}\cite{puppeteer}.

	\section{Rumusan Masalah}
       	Rumusan masalah yang diangkat dalam tugas akhir ini dapat dipaparkan sebagai berikut :
		\begin{enumerate}
			\item Bagaimana cara mengimplementasikan \textit{Headless Chrome} sebagai \textit{tester} untuk \textit{load generator}?
			\item Bagaimana cara mengimplementasikan \textit{Docker} bisa menjadi \textit{load generator} untuk uji beban?
			\item Bagaimana cara menghasilkan laporan uji beban pada web?
			\item Bagaimana mengelola layanan pengujian untuk \textit{multiuser} dalam bentuk antrian?
		\end{enumerate}

	\section{Batasan Masalah}
		Dari permasalahan yang telah dipaparkan di atas, terdapat beberapa batasan masalah pada tugas akhir ini, yaitu:
		\begin{enumerate}
			\item \textit{Headless Browser} yang digunakan adalah \textit{Headless Chrome}.
			\item Kontainer yang digunakan adalah \textit{Docker}.
			\item Distribusi kontainer menggunakan \textit{Docker Swarm}.
			\item Aplikasi yang akan diuji berupa aplikasi web.
			\item Uji coba aplikasi akan menggunakan  \textit{Node library yang menyediakan API (Application Programming Interface)} untuk mengontrol \textit{Headless Chrome} yaitu \textit{Puppeteer}.
		\end{enumerate}

	\section{Tujuan}
       	Tujuan pembuatan tugas akhir ini antara lain:
       	\begin{enumerate}
       		\item Membuat sistem manajemen pengujian aplikasi secara \textit{headless} menggunakan \textit{Headless Chrome}.
       		\item Membuat sistem agar \textit{Docker} bisa menjadi \textit{load generator} untuk pengujian.
       		\item Membuat sistem untuk menampilkan laporan uji beban pada web.
       		\item Membuat sistem yang dapat mengelola permintaan layanan uji beban dari \textit{multiuser} berupa antrian.
       	\end{enumerate}
        
	\section{Manfaat}
		Manfaat dari pembuatan tugas akhir ini yaitu:
		\begin{enumerate}
			\item Mempelajari penggunaan \textit{Headless Browser} untuk pengujian suatu aplikasi yaitu \textit{Headless Chrome}
			\item Mempelajadi penggunaan \textit{Docker} sebagai \textit{load generator} untuk melakukan uji beban.
			\item Mengetahui performa uji beban suatu aplikasi web.
			\item Meminimalisir adanya kegagalan web saat dimuat ketika sudah diluncurkan.
		\end{enumerate}
	
	\section{Metodologi}
		Metodologi yang digunakan untuk pembuatan Tugas Akhir ini adalah sebagai berikut:	
		\subsection{Penyusunan Proposal Tugas Akhir}
			Proposal tugas akhir ini berisi tentang deskripsi pendahuluan dari tugas
			akhir yang akan dibuat. Pendahuluan ini terdiri dari hal yang menjadi latar
			belakang diajukannya usulan tugas akhir, rumusan masalah yang diangkat,
			batasan masalah untuk tugas akhir, tujuan dari pembuatan tugas akhir, dan
			manfaat dari hasil pembuatan tugas akhir. Selain itu dijabarkan pula tinjauan
			pustaka yang digunakan sebagai referensi pendukung pembuatan tugas akhir.
			Sub bab metodologi berisi penjelasan mengenai tahapan penyusunan tugas
			akhir mulai dari penyusunan proposal hingga penyusunan buku tugas akhir.
			Terdapat pula sub bab jadwal kegiatan yang menjelaskan jadwal pengerjaan
			tugas akhir.	
		\subsection{Studi Literatur}
			Pada tahap ini dilakukan pencarian informasi dan referensi mengenai \textit{Headless Chrome}, \textit{Puppeteer} dan \textit{Docker} untuk mendukung dan memastikan setiap tahap pembuatan tugas akhir sesuai dengan standar dan konsep yang berlaku, serta dapat diimplementasikan. Sumber informasi dan referensi bisa didapatkan dari buku, jurnal dan internet.
		\subsection{Analisis dan Desain Perangkat Lunak}
			Pada tahap ini dilakukan analisis dan perancangan terhadap arsitektur sistem yang akan dibuat. Tahap ini merupakan tahap yang paling penting dimana segala bentuk implementasi bisa bekerja dengan baik ketika arsitektur sistem yang baik pula.
		\subsection{Implementasi Perangkat Lunak}
			Pada tahap ini dilakukan implementasi atau realisasi dari hasil analisis dan perancangan arsitektur yang sudah dibuat sebelumnya, sehingga menjadi sebuah infrastruktur yang sesuai dengan apa yang direncanakan. 
		\subsection{Pengujian dan Evaluasi}
			Pada tahap ini dilakukan pengujian untuk mengukur performa web dan kegagalan saat dimuat menggunakan arsitektur sistem yang sudah dibuat menggunakan infrastruktur \textit{Docker}. Beberapa performa yang diukur pada pengujian antara lain, \textit{load time}, \textit{response time}, \textit{firstmeaningfulpain}, \textit{css tracing} dan \textit{domcontentloadevent} dalam satuan \textit{ms(millisecond)} serta \textit{error console}. Setelah dilakukan uji coba, maka dilakukan evaluasi terhadap kinerja arsitektur sistem yang telah diimplementasikan dengan harapan bisa diperbaiki ketika ada pengembangan ke depannya.
		\subsection{Penyusunan Buku Tugas Akhir}
			Pada tahap ini dilakukan penyusunan buku tugas akhir yang berisikan dokumentasi yang mencakup teori, konsep, implementasi dan hasil pengerjaan tugas akhir.
	
	\section{Sistematika Penulisan}
		Sistematika penulisan laporan tugas akhir secara garis besar adalah sebagai berikut:
		\begin{enumerate}
			\item Bab I. Pendahuluan\\
				Bab ini berisi penjelasan mengenai latar belakang, rumusan masalah, batasan masalah, tujuan, manfaat, metodologi dan sistematika penulisan dari pembuatan tugas akhir.
			\item Bab II. Tinjauan Pustaka\\
				Bab ini berisi kajian teori atau penjelasan metode, algoritme, \textit{library} dan \textit{tools} yang digunakan dalam penyusunan tugas akhir ini. Kajian teori yang dimaksud berisi tentang penjelasan singkat mengenai \textit{Headless Browser}, \textit{Headless Chrome}, \textit{Puppeteer}, \textit{NodeJS}, \textit{Docker}, \textit{Docker Swarm}, \textit{Laravel} dan \textit{Python}.
			\item Bab III. Desain dan Perancangan\\
				Bab ini berisi mengenai analisis dan perancangan arsitektur sistem yang akan diimplementasikan pada tugas akhir ini.
			\item Bab IV. Implementasi\\
				Bab ini berisi bahasan tentang implementasi dari arsitektur sistem yang dibuat pada bab sebelumnya. Penjelasan berupa kode program yang digunakan untuk mengimplementasikan sistem.
			\item Bab V. Pengujian dan Evaluasi\\
				Bab ini berisi bahasan tentang tahapan uji coba terhadap performa web dan evaluasi terhadap sistem yang dibuat.
			\item Bab VI. Penutup\\
				Bab ini merupakan bab terakhir yang memaparkan kesimpulan dari hasil pengujian dan evaluasi yang telah dilakukan. Pada bab ini juga terdapat saran yang ditujukan bagi pembaca yang berminat untuk melakukan pengembangan lebih lanjut.
			\item Daftar Pustaka\\
				Bab ini berisi daftar pustaka yang dijadikan literatur dalam tugas akhir.
			\item Lampiran\\
				Dalam lampiran terdapat kode sumber program secara keseluruhan.
		\end{enumerate}
	\chapter{TINJAUAN PUSTAKA}
	\section{\textit{Headless Browser}}
		\textit{Headless Browser} merupakan jenis perangkat lunak yang dapat mengakses halaman web, namun tidak menampilkan antarmuka pengguna. Seperti peramban pada umumnya, \textit{Headless Browser} juga memiliki kemampuan yang sama, hanya saja menyisakan mesin dan lingkungan \textit{javascript}. Oleh karena itu \textit{Headless Browser} hanya bisa diakses menggunakan baris perintah atau \textit{CLI(Command Line Interface)}\cite{headless_browser}. Beberapa \textit{Headless Browser} yaitu \textit{Headless Chrome}, \textit{Selenium WebDriver} dan \textit{Firefox Headless Mode}. Salah satu \textit{Headless Browser} yang akan digunakan pada Tugas Akhir ini adalah \textit{Headless Chrome}, karena \textit{Headless Chrome} memiliki fitur khusus yang bisa ditemukan pada bagian \textit{performance}. \textit{Headless Chrome} juga memiliki fitur umum seperti \textit{Headless Browser} yang lain. Beberapa fitur umumnya yaitu\cite{headless_browser_2}:
		
		\begin{enumerate}
			\item Memudahkan untuk melakukan pengujian secara otomatis.
			\item Bisa dijalankan di server, mode \textit{headless} tidak membutuhkan antarmuka pengguna.
			\item Membuat dokumen atau file seperti PDF dan Screenshot.
			\item \textit{Debugging}.\\
		\end{enumerate}

		\indent Dalam tugas akhir ini, \textit{Headless Browser} akan digunakan sebagai browser untuk menguji web yang akan diuji performanya, sedangkan jenis yang digunakan adalah \textit{Headless Chrome}. \textit{Headless Chrome} saat ini dikembangkan oleh \textit{Google Developer} dan memiliki lisensi dari \textit{Apache 2.0 License}.
			
	\section{\textit{Node.js}}
		\textit{Node.js} atau \textit{node} adalah sebuah platform dengan lingkungan \textit{JavaScript} sisi server.\textit{ Node.js} berbasis pada \textit{Chrome's JavaScript Runtime} yang menggunakan teknologi V8 dan berfokus pada performa maupun konsumsi memori rendah. Tapi V8 juga mendukung proses server yang berjalan lama. Tidak seperti kebanyakan platform modern yang lain dengan mengandalkan \textit{multithreading}. \textit{Node.js} menggunakan penjadwalan I/O secara asinkron. Proses pada \textit{Node.js} dibayangkan sebagai proses \textit{single-threaded daemon}. Hal ini berbeda dengan kebanyakan sistem penjadwalan dalam bahasa pemrograman lain yang berbentuk \textit{library}. \textit{Node.js} seringkali digunakan pengembang sebagai web server atau layanan \textit{API}. Selain itu \textit{Node.js} juga mendukung \textit{event callback} untuk setiap penggunaan fungsi, memungkinkan ketika fungsi tersebut dipanggil akan terjadi \textit{sleep} ketika tidak ada hasil apapun.\cite{nodejs}\cite{nodejs_2}.
		
	 	\indent Salah satu pustaka \textit{Node.js} yang menyediakan layanan \textit{API} adalah \textit{Puppeteer}. Pada tugas akhir ini, \textit{Node.js} akan digunakan sebagai bahasa pemrograman untuk mengimplementasikan \textit{Puppeteer}.
		
		\subsection{\textit{Puppeteer}}
			\textit{Puppeteer} adalah sebuah pustaka dari \textit{Node} yang memiliki kemampuan yang mumpuni untuk memberikan layanan \textit{API} yang berfungsi untuk mengontrol layanan dari \textit{Chrome} atau \textit{Chromium}. Selain itu, kemampuan kontrol \textit{Puppeteer} sangat memungkinkan untuk akses pada protokol \textit{Devtools Protocol} yang saat ini dikembangkan oleh tim \textit{Google Developer}, dimana protokol tersebut memiliki kemampuan yang cukup berguna yaitu sebagai \textit{tools instrument}, \textit{inspect}, \textit{debug}, dan \textit{profile chrome}.\cite{puppeteer} \\
			\indent Dibandingkan dengan \textit{PhantomJS} yang sudah tidak dikembangkan lagi, \textit{Puppeteer} masih dikembangkan secara berkala. Begitupun fitur-fitur yang disediakan \textit{Puppeteer} sangat mumpuni untuk melakukan beberapa pengujian terhadap web yaitu melakukan pengambilan gambar ataupun pdf, otomasi, pengujian antarmuka pengguna, \textit{keyboard input}, \textit{timeline trace} untuk mendapatkan performa dan ekstensi pada \textit{Chrome}. Untuk lebih jelasnya struktur diagram \textit{Puppeteer} ditunjukkan pada Gambar \ref{puppeteeroverview}.
			\begin{figure}[H]
				\centering
				\includegraphics[width=9cm,height=10cm]{Images/C-2/puppeteeroverview.jpg}
				\caption{Struktur diagram \textit{Puppeteer}}
				\label{puppeteeroverview}
			\end{figure}
			
			\indent Pada tugas akhir ini, \textit{Puppeteer} akan digunakan sebagai alat untuk mengontrol \textit{Headless Browser} yang akan diimplementasikan pada sistem perangkat lunak yang akan dibangun, karena lebih mumpuni dan memiliki beragam fitur \textit{API} untuk mengakses \textit{Headless Chrome} dibandingkan dengan pustaka \textit{Node} yang lain.
		
	\section{\textit{Docker}}
		\textit{Docker} adalah sebuah platform terbuka yang berfungsi sebagai wadah untuk membangun, membungkus, dan menjalankan aplikasi supaya dapat berfungsi sebagaimana mestinya. \textit{Docker} memungkinkan untuk memisahkankan aplikasi dari infrastruktur supaya \textit{software} dapat di jalankan dengan lebih cepat. \textit{Docker} pada dasarnya memperluas \textit{LXC(Linux Containers)} menggunakan kernel dan \textit{API} pada level aplikasi yang akan dijalankan secara bersamaan pada isolasi \textit{CPU}, memori, I/O, jaringan dan yang lainnya. \textit{Docker} juga menggunakan \textit{namespaces} untuk mengisolasi segala tampilan pada aplikasi yang mendasari lingkungan operasinya, termasuk \textit{process tree}, jaringan, ID pengguna, dan file sistem. 
		
		\indent \textit{Docker Container} dibuat oleh sebuah \textit{Docker Images}. \textit{Docker Images} hanya mencakup dasar dari operasi sistem atau hanya memuat set dari \textit{prebuilt} aplikasi yang sudah siap dijalankan. Ketika membuat \textit{Docker Images}, bisa menjalankan perintah (yaitu apt-get install) membentuk lapisan baru diatas lapisan sebelumnya. Perintah tersebut bisa dijalankan manual satu-persatu atau secara otomatis menggunakan \textit{Dockerfile}. 
		
		\indent Setiap \textit{Dockerfile} adalah kombinasi beberapa perintah yang dibuat menjadi menjadi satu atau \textit{script} yang bisa dijalankan secara otomatis sebagai \textit{Docker Images} utama atau untuk membuat \textit{Docker Images} yang baru\cite{docker}\cite{docker_2}. \textit{Docker} juga menyediakan layanan untuk mengunduh dan mengupload \textit{Docker images} melalui \texttt{https://hub.docker.com/}. Untuk melihat perbedaan antara kontainer dan \textit{VM(Virtual Machine)} dapat dilihat pada Gambar \ref{containervm}.
	
		\indent Pada tugas akhir ini, \textit{Docker Container} akan digunakan sebagai pengguna untuk mengakses web yang akan diuji, dan bisa diumpamakan sebagai pengguna asli untuk otomasi pengujiannya. \\
		
		\begin{figure}[H]
			\centering
			\includegraphics[width=9cm,height=7cm]{Images/C-2/containervm.png}
			\caption{Perbandingan kontainer dan \textit{Virtual Machine}\cite{docker_2}}
			\label{containervm}
		\end{figure}
	
		\subsection{\textit{Docker Swarm}}
			\textit{Docker Swarm} - disebut juga \textit{Swarm} adalah mode pada \textit{Docker} yang memiliki fitur yang tertanam pada mesinnya untuk Manajemen \textit{Cluster} atau \textit{Orkestrasi}. Pada mode \textit{Swarm} akan terdapat lebih dari satu \textit{host} dimana \textit{host} tersebut bisa berfungsi sebagai manager, worker, atau bisa juga keduanya. Konsep pada \textit{Swarm} antara lain adalah \textit{Nodes}, \textit{Services}, \textit{Tasks} dan \textit{Load Balancing}. Mode \textit{Swarm} juga memudahkan untuk mengatur bagian replikasi, jaringan, penyimpanan, port dan sebagainya.
			
			\indent Dibandingkan dengan kontainer yang berdiri sendiri, \textit{Swarm} lebih mudah untuk mengubah konfigurasi servis, termasuk penyimpanan dan jaringannya tanpa harus menyalakan kembali kontainer secara manual. \textit{Docker} akan otomatis memperbarui konfigurasi dengan cara menghentikan \textit{service task} yang memiliki konfigurasi lama, kemudian akan membuat kembali \textit{service task} menggunakan konfigurasi yang sudah diperbarui. \textit{Swarm} juga bisa menggunakan \textit{Docker Compose} untuk mendefinisikan dan menjalankan kontainer, \textit{Docker Compose} menggunakan \textit{YAML file} sebagai konfigurasinya.\cite{docker_swarm}.
			
			\indent Pada saat mode \textit{Swarm}, \textit{node manager} akan mengimplementasikan \textit{Raft Consensus Algorithm} untuk memanajemen \textit{cluster}. \textit{Consensus} memungkinan manajer untuk mengatur dan menjadwal \textit{tasks} pada setiap \textit{cluster} dan memastikan status tetap konsisten, dimana ketika ada salah satu \textit{nodes} yang gagal dalam mejalankan servis, manajer bisa mengembalikan servis menjadi stabil kembali\cite{docker_swarm_raft}. Untuk melihat rute diagram ketika mode \textit{Swarm} dapat dilihat pada Gambar \ref{swarmdiagramroute}.
			
			\begin{figure}[H]
				\centering
				\includegraphics[width=10cm,height=4cm]{Images/C-2/swarmdiagramroute.png}
				\caption{Rute diagram ketika mode \textit{Swarm}\cite{docker_swarm_route}}
				\label{swarmdiagramroute}
			\end{figure}
		
			\indent Pada tugas akhir ini, \textit{Docker Swarm} akan digunakan sebagai manager atau okestrasi yang mengatur segala servis maupun aktivitas \textit{Docker Container} dan sebagai \textit{Load Balancer} untuk pembagian beban kontainer pada setiap \textit{nodes} yang merupakan instansi dari \textit{Docker Swarm} tersebut.
			
			
		
	\section{\textit{Laravel}}
		\textit{Laravel} adalah salah satu kerangka kerja yang berbahasa \textit{PHP} dan dibuat untuk memudahkan pengembang untuk mengembangan dan mendesain sebuah web yang menekankan kesederhanaan dan fleksibitas. Kerangka kerja ini mendukung metode \textit{MVC(Model-View-Controller)}. dimana \textit{MVC} digunakan untuk mengembangkan sebuah aplikasi yang memisahkan data(\textit{Model}) dari tampilan(\textit{View}) dan juga dari logika dari aplikasi tersebut(\textit{Controller})\cite{laraveframework}.
		
		\indent \textit{Model} digunakan untuk memanipulasi data dari basis data, \textit{View} berhubungan dengan antarmuka web seperti \textit{HTML}, \textit{CSS} dan \textit{JS} sebagai data pada pengguna. \textit{Controller} berhubungan dengan segala urusan logika pada servis web tersebut atau juga bisa disebut otaknya. \textit{Controller} juga berfungsi sebagai jembatan antara \textit{View} dan \textit{Model}\cite{laraveframework}.
	
		\indent Pada tugas akhir ini, kerangka kerja \textit{Laravel} akan digunakan untuk mengimplementasikan aplikasi web yang dibangun pada tugas akhir ini, dimana kerangka kerja ini sangat banyak digunakan oleh pengembang, memiliki dokumentasi resmi yang sangat baik, serta forum yang cukup baik. \textit{Laravel} yang akan digunakan adalah versi 5.8.
		
	\section{\textit{Python}}
		\textit{Python} adalah bahasa pemrograman tingkat tinggi yang didukung oleh struktur data \textit{built-in} semantik dinamis, selain itu Python mendukung pemrograman \textit{procedural}, \textit{object-oriented} dan \textit{functional}. \textit{Python} merupakan bahasa pemrograman \textit{interpreted}, oleh sebab itu, \textit{Python} tidak memakan biaya untuk kompilasi, sehingga proses pengembangan, pengujian dan \textit{debug} menjadi lebih cepat.
		 
		\indent Kelebihan bahasa pemrograman ini adalah memiliki modul dan \textit{package}, serta memiliki banyak standar pustaka yang didistribusikan secara bebas dan gratis. Selain itu \textit{Python} mudah dibaca karena memiliki sintaksis yang sederhana, sehingga dapat mengurangi biaya \textit{maintenance}. \textit{Debug} pada \textit{Python} juga mudah karena tidak akan terjadi \textit{segmentation fault}, namun akan memberi umpan balik berupa \textit{exception} apabila terdapat kesalahan atau \textit{error}\cite{python}.
		
		\indent Pada tugas akhir ini, Bahasa pemrograman \textit{Python} akan digunakan untuk mengimplementasikan algoritme \textit{task queue} pada sistem yang akan dibangun. \textit{Python} yang digunakan adalah versi 3.6.8.
	\chapter{DESAIN DAN PERANCANGAN}
    Pada bab ini dibahas mengenai analisis dan perancangan sistem.
    
    \section{Deskripsi Umum Sistem}
    	\indent Sistem yang akan dibangun pada tugas akhir ini adalah sebuah sistem yang dapat melakukan automasi uji beban terhadap suatu web. Uji beban pada sistem akan berjalan secara headless menggunakan sebuah tester yaitu Headless Chrome. Headless Chrome akan mendapatkan data uji beban ketika mengakses web yang diuji, sedangkan yang digunakan untuk mengambil data uji beban adalah sebuah pustaka Node yaitu Puppeteer. Sistem juga akan menggunakan Docker sebagai infrastruktur, sehingga Docker dapat digunakan sebagai load generator untuk melakukan uji beban yang bisa disebut kontainer.
    	
    	\indent Kontainer yang akan dipasang pada sistem membutuhkan sebuah alat orkestrasi untuk memanajemen kontainer secara otomatis, alat orkestrasi yang digunakan adalah Docker Swarm. Docker Swarm akan melibatkan 3 node host yang akan dibagi menjadi 1 node host sebagai swarm manager dan 2 node host sebagai worker. Docker Swarm akan bertanggung jawab dalam mendistribusikan kontainer ke masing-masing swarm node yang tergabung pada lingkungan swawrm atau bisa disebut sebagai load balancer.
    	
    	\indent Proses uji beban akan diproses user melakukan request skenario uji beban pada web service yang disediakan sistem. Kemudian controller pada web service akan mengirimkan skenario uji pada kontainer terpilih untuk melakukan pengujian. Setiap kontainer akan terinstall Headless Chrome dan Puppeteer. Headless Chrome akan mendapatkan data uji beban dan automasi pengambilan data uji beban dilakukan oleh Puppeteer. Puppeteer akan melakukan ekstraksi data uji beban menjadi satuan millisecond(ms). 
    	
    	\indent Sistem akan menyediakan basis data untuk menyimpan data yang diperlukan sistem. Basis data akan dipasang diluar lingkungan swarm dan lingkungan web service. Sistem juga menyediakan antarmuka pengguna berupa web yang akan digunakan untuk melihat laporan hasil uji beban. Sedangkan untuk mengatasi multiuser sistem akan menggunakan Task Scheduler atau Queue(antrian).
    
    \section{Kasus Penggunaan}
	    \begin{figure}[H]
	    	\centering
	    	\includegraphics[width=9cm,height=9cm]{Images/C-3/usecasediagram.png}
	    	\caption{Diagram kasus penggunaan}
	    	\label{usecased}
	    \end{figure}
    	Terdapat dua aktor dalam sistem yang akan dibuat yaitu User dan Kontainer. User merupakan aktor yang bisa melakukan manajemen pada skenario yang ingin diuji dan melihat hasilnya, sedangkan Kontainer merupakan aktor yang akan digunakan sebagai load generator untu melakukan uji beban. Diagram kasus penggunaan digambarkan pada Gambar \ref{usecased} dan dijelaskan masing-masing pada Table \ref{tabelusecase}.
    	
    	\begin{longtable}{|p{0.20\textwidth}|p{0.30\textwidth}|p{0.35\textwidth}|}
    		\caption{Daftar kode kasus penggunaan} \label{tabelusecase} \\
    		\hline
    		\textbf{Kode Kasus Penggunaan} & \textbf{Nama Kasus Penggunaan} & \textbf{Keterangan} \\ \hline
    		\endhead
    		\endfoot
    		\endlastfoot
    		UC-0001 & Manajemen Skenario Uji Beban & User dapat menambah, melihat dan menghapus skenario uji beban \\ \hline
    		UC-0002 & Mengirim Permintaan Uji Beban & User dapat mengirimkan permintaan uji beban ke sistem melalui web service yang disediakan \\ \hline
    		UC-0003 & Melihat Hasil Uji Beban & Ketika proses uji beban selesai, user dapat melihat hasilnya di antarmuka pengguna web service yang disediakan \\ \hline
    		UC-0004 & Menerima Permintaan Uji Beban & Proses dimana kontainer akan menerima permintaan uji beban dari User \\ \hline
    		UC-0005 & Melakukan Uji Beban & Proses dimana kontainer akan melakukan uji beban sesuai skenario yang dikirim \\ \hline
    		UC-0006 & Mengirim Hasil Uji Beban ke Basis Data & Ketika kontainer telah selesai melakukan pengujian, data yang didapatkan akan dikirim ke basis data MySQL \\ \hline
    	\end{longtable}
    
    \section{Arsitektur Sistem}
	    \begin{figure}[H]
	    	\centering
	    	\includegraphics[width=10cm,height=6cm]{Images/C-3/arsitektursistem.png}
	    	\caption{Desain arsitektur sistem}
	    	\label{arsitekturumum}
	    \end{figure}
    	
    	\indent Pada sub-bab ini, akan dibahas mengenai tahap analisis arsitektur, analisis teknologi dan desain sistem yang akan dibangun. Arsitektur sistem secara umum ditunjukkan pada Gambar \ref{arsitekturumum}.

    	\subsection{Desain Umum Sistem}
    		Berdasarkan yang dijelaskan pada deskripsi umum sistem, dapat diperoleh beberapa kebutuhan sistem antara lain:
    		\begin{enumerate}
    			\item Load generator untuk melakukan uji beban.
    			\item Tester yang bisa mengambil data uji beban.
    			\item Web service sebagai antarmuka pengguna.
    			\item Basis data untuk menyimpan data sistem.
    			\item Task Queue untuk menangani kasus request lebih dari satu user.
    		\end{enumerate}
    	
    		Untuk memenuhi kebutuhan sistem yang dijelaskan sebelumnya, penulis membagi menjadi beberapa komponen sistem yang akan digunakan pada tugas akhir ini.
    		
    		\begin{enumerate}
    			\item Load generator \\
    				Berfungsi sebagai pengganti user yang akan melakukan akses web melalui browser.
    			\item Pengambil data uji beban \\
    				Berfungsi untuk mengambil data uji beban ketika load generator mengakses web dari browser.
    			\item Service Controller \\
    				Berfungsi sebagai pengatur sistem uji beban yang terdiri :
    				\begin{itemize}
    					\item Web Service \\
    						Berfungsi sebagai tampilan antarmuka pengguna untuk menggunakan sistem.
    					\item Basis Data \\
    						Berfungsi untuk menyimpan data yang digunakan untuk menyimpan segala data yang dibutuhkan oleh sistem.
    					\item Task Queue \\
    						Berfungsi untuk membuat task scheduler atau antrian untuk menangani kasus request lebih dari satu user.
    				\end{itemize}
    		\end{enumerate}
    	
    	\subsection{Perancangan Load Generator}
	    	\begin{figure}[h]
	    		\centering
	    		\includegraphics[width=11cm,height=6cm]{Images/C-3/dockerdesain.png}
	    		\caption{Desain perancangan load generator}
	    		\label{dockerdesain}
	    	\end{figure}
    		Komponen load generator akan difungsikan sebagai pengganti pengguna yang mengakses ke suatu web melalui browser. Komponen yang akan digunakan sebagai load generator adalah Docker atau bisa disebut kontainer. Ketika melakukan pemasangan kontainer sebuah Docker Image, untuk memenuhi hal tersebut, pada tugas akhir ini penulis akan membuat sebuah Docker Image yang akan diunggah ke Docker Hub. Sehingga ketika akan memasang kontainer pada node host yang baru, node host tersebut hanya perlu mengunduh Docker Image yang telah diunggah sebelumnya. Seluruh kontainer akan dibangun di dalam lingkungan swarm untuk memudahkan dalam mengatur atau memanajemen kontainer ke semua node host yang tergabung di dalam lingkungan swarm, sedangkan untuk memudahkan akses ke setiap kontainer, maka data dari kontainer akan disimpan di dalam basis data MySQL. Load generator akan terdiri dari 3 node host, 1 sebagai manager node dan 2 lainnya sebagai worker, sedangkan basis data akan berada di luar lingkungan swarm. Desain perancangan komponen ini digambarkan pada Gambar \ref{dockerdesain}.
    		 
    	
    	\subsection{Perancangan Pengambil Data Uji Beban}
    		Komponen ini akan membutuhkan suatu alat yang bisa mendapatkan data uji beban terlebih dahulu. Pada tugas akhir ini, akan menggunakan Headless Chrome untuk mendapatkan data uji beban ketika mengakses web. Setelah mendapatkan data uji beban, diperlukan juga suatu alat yang bisa digunakan untuk mengambil data uji beban pada Headless Chrome, alat tersebut adalah Puppeteer. Puppeteer akan melakukan pengambilan secara otomatis ketika ada perintah yang masuk dan menyimpan data uji beban pada basis data MySQL. Alat-alat yang digunakan pada komponen ini akan dipasang pada masing-masing kontainer di setiap node host. Desain perancangan komponen ini digambarkan pada Gambar \ref{puppdesain}.
    		\begin{figure}[H]
    			\centering
    			\includegraphics[width=10cm,height=4.5cm]{Images/C-3/puppdesain.png}
    			\caption{Desain pengambil data uji beban}
    			\label{puppdesain}
    		\end{figure}
    	
    	\subsection{Perancangan Service Controller}
    		Komponen ini akan digunakan untuk mengatur segala proses uji beban pada sistem. Pada komponen ini akan terdapat 3 buah sub-komponen yaitu web service, basis data dan task queue atau antrian.
    	
	    	\subsubsection{Desain Web Service}
	    		Web service akan berfungsi sebagai antarmuka pengguna dan sebagai penghubung antara user dengan kontainer. Antarmuka pengguna berfungsi memudahkan user untuk membuat skenario yang akan dikirimkan ke load generator dan kemudian load generator akan melakukan uji beban sesuai dengan skenario yang dikirim user melalui web service. Sedangkan untuk mengatur segala aktivitas user dibutuhkan sebuah controller dan rute yang akan dipasang pada web service. Desain antarmuka pengguna ditunjukkan pada Gambar \ref{mockupweb}.
	    		\begin{figure}[H]
	    			\centering
	    			\includegraphics[width=10cm,height=6cm]{Images/C-3/mockupweb.png}
	    			\caption{Desain antarmuka pengguna}
	    			\label{mockupweb}
	    		\end{figure}
	    	
		    	Selain itu, akan dirancang juga fitur-fitur pada web service yang akan digunakan user antara lain:
		    	\begin{enumerate}
		    		\item Menambah dan menghapus skenario.
		    		\item Menentukan jumlah worker pengujian.
		    		\item Melihat performa hasil pengujian.
		    		\item Melihat tangkapan layar tampilan web yang diuji.
		    		\item Melihat console error.
		    		\item Melihat status antrian proses uji. \\
		    	\end{enumerate}
	    		
	   		\subsubsection{Desain Basis Data}
	   			Komponen basis data diperlukan untuk menyimpan data-data yang berkaitan dengan sistem. data yang disimpan adalah data node host swarm, data kontainer, data pengguna, data skenario pengujian, data antrian request, data hasil pengujian, data error console, data rata-rata hasil pengujian. Dari data-data tersebut maka dibutuhkan suatu tabel diantaranya yaitu:
	   			\begin{itemize}
	   				\item Tabel swarms \\
	   					Menyimpan data nohe host yang tergabung di dalam lingkungan swarm.
	   				\item Tabel containers \\
	   					Menyimpan data Docker Container yang telah dipersiapkan.
	   				\item Tabel users \\
	   					Menyimpan data pengguna.
	   				\item Tabel scenarios \\
	   					Menyimpan data skenario pengujian.
	   				\item Tabel queues \\
		   				Menyimpan data antrian request pengujian dari user.
	   				\item Tabel results \\
	   					Menyimpan data hasil pengujian yang dilakukan setiap kontainer.
	   				\item Tabel errors \\
	   					Menyimpan data console error yang ada di browser.
	   				\item Tabel summary results \\
	   					Menyimpan data rata-rata hasil pengujian setiap skenario.
	   			\end{itemize}
	    		
	    	\subsubsection{Desain Penggunaan Task Queue}
	    		Pada service controller akan ada banyak request dari user, setiap request tentu saja akan terdapat proses yang akan berjalan dalam jangka waktu yang cukup lama. Jika proses tersebut berada di dalam fungsi yang dipanggil melalui protokol HTTP, maka akan memberikan umpan balik setelah semua proses yang ada dibaliknya selesai. Hal ini akan membuat user yang melakukan request perlu menunggu dan tidak efisien. Untuk mengatasi hal ini, akan dirancang sebuah komponen antrian atau bisa disebut task queue. Task queue akan membuat antrian untuk setiap request dibelakang layar. Antrian request tersebut akan disimpan pada basis data MySQL.
	\chapter{IMPLEMENTASI}
	Bab ini membahas mengenai implementasi dari sistem yang sudah di desain dan dirancang pada bab sebelumnya. Pembahasan secara rinci akan dijelaskan pada setiap komponen yang ada yaitu load generator, pengambil data uji beban, web service dan task queue.
	
	\section{Lingkungan Implementasi}
		Dalam mengimplementasikan sistem, digunakan beberapa perangkat pendukung sebagai berikut.
		
		\subsection{Perangkat Keras}
		Perangkat keras yang digunakan dalam pengembangan sistem adalah sebagai berikut:
		\begin{enumerate}
			\item Web service dan task queue, processor AMD FX-7600P Radeon R7, 12 Compute Cores 4C+8G dan RAM 8GB.
			\item Node host docker swarm dengan IP 167.71.194.235, processor Intel(R) Xeon(R) CPU E5-2650 v4 @ 2.20GHz dan RAM 4GB.
			\item Node host docker swarm dengan IP 165.22.55.82, processor Intel(R) Xeon(R) CPU E5-2650 v4 @ 2.20GHz dan RAM 4GB.
			\item Node host docker swarm dengan IP 167.71.194.233, processor Intel(R) Xeon(R) CPU E5-2650 v4 @ 2.20GHz dan RAM 4GB.
			\item Basis data MySQL dengan IP 178.128.123.143, processor Intel(R) Xeon(R) Gold 6140 CPU @ 2.30GHz dan RAM 1GB.
		\end{enumerate}
	
		\subsection{Perangkat Lunak}
		Perangkat lunak yang digunakan dalam pengembangan sistem adalah sebagai berikut:
		\begin{enumerate}
			\item Sistem Operasi Ubuntu 18.04 LTS 64 Bit
			\item Docker versi 18.09.6 
			\item Headless Chrome 
			\item Puppeteer versi 0.12.0
			\item NPM versi 6.4.1
			\item Node.js versi 8.15.1
			\item Python versi 3.6.8
			\item MySQL Ver 14.14 Distrib 5.7.26
			\item Shell Script
			\item PHP dan Laravel 5.8
		\end{enumerate}
		
	\section{Implementasi Load Generator}
		Berdasarkan perancangan dan desain, load generator merupakan aktor yang akan berfungsi menggantikan pengguna ketika mengakses ke web. Load generator yang akan digunakan adalah Docker, namun diperlukan beberapa tahap untuk bisa menggunakan Docker atau kontainer sebagai load generator, yaitu tahap pemasangan dan konfigurasi. Tahap pemasangan docker dapat dilihat di Kode Sumber \ref{instalasidocker}, sedangkan untuk konfigurasi akan dibagi menjadi beberapa tahap yaitu:
		\begin{enumerate}
			\item Pembuatan Docker Image
			\item Pembuatan lingkungan kontainer
			\item Pemasangan Headless Chrome dan Puppeteer
		\end{enumerate}
			
		\subsection{Implementasi Pembuatan Docker Image}
			Docker Image digunakan untuk menjalankan kontainer, pada tugas akhir ini Docker Image akan dibuat terlebih dahulu agar bisa digunakan untuk menjalankan Puppeteer dan Headless Chrome. Namun untuk membuat Docker Image diperlukan beberapa tahapan yaitu konfigurasi Dockerfile, pemasangan Docker Compose dapat dilihat pada Kode Sumber \ref{dockercomposeinstall} dan konfigurasi docker-compsoe.yml dan unggah Docker Image ke Docker Hub.
			
			\indent Konfigurasi Dockerfile dapat dilihat pada Kode Sumber \ref{dockerimage}, sedangkan konfigurasi docker-compose.yml dapat dilihat pada Kode Sumber \ref{dockercompose}.
				\begin{lstlisting}[frame=single,tabsize=2,breaklines,caption={Konfigurasi Dockerfile },label=dockerimage, captionpos=b, language=json]
	FROM node:8
	RUN apt-get update
	# for https
	RUN apt-get install -yyq ca-certificates
	# install libraries
	RUN apt-get install -yyq libappindicator1 libasound2 libatk1.0-0 libc6 libcairo2 libcups2 libdbus-1-3 libexpat1 libfontconfig1 libgcc1 libgconf-2-4 libgdk-pixbuf2.0-0 libglib2.0-0 libgtk-3-0 libnspr4 libnss3 libpango-1.0-0 libpangocairo-1.0-0 libstdc++6 libx11-6 libx11-xcb1 libxcb1 libxcomposite1 libxcursor1 libxdamage1 libxext6 libxfixes3 libxi6 libxrandr2 libxrender1 libxss1 libxtst6
	# tools
	RUN apt-get install -yyq gconf-service lsb-release wget xdg-utils
	RUN apt-get install -yyq fonts-liberation 
	COPY code /app/code
	COPY output /app/output
	WORKDIR /app/code
	RUN yarn install
				\end{lstlisting}
				
				\begin{lstlisting}[frame=single,tabsize=2,breaklines,caption={Konfigurasi docker-compose.yml },label=dockercompose, captionpos=b, language=json]
	version: '3'
	services:
		puppeteer:
			build: .
			shm_size: '1gb'
			entrypoint: ["sh", "-c", "sleep infinity"]
				\end{lstlisting}
				
				Kemudian jalankan perintah Docker Compose berikut untuk pembuatan Docker Image.
				\begin{lstlisting}[frame=single,tabsize=2,breaklines,caption={Perintah untuk menjalankan  Docker Compose},label=createimg, captionpos=b, language=json,numbers=none]
	$ docker-compose up
				\end{lstlisting}
				
				Setelah Docker Image terbuat, ubah nama Docker Image tersebut dan melakukan commit agar bisa di push. Perintah pada kode sumber \ref{pushimagedocker} yang digunakan oleh penulis ketika mengunggah Docker Image ke Docker Hub.
				\begin{lstlisting}[frame=single,tabsize=2,breaklines,caption={Perintah untuk mengunggah Docker Image},label=pushimagedocker, captionpos=b, language=json,numbers=none]
	$ docker tag puppeteer_puppeteer:latest cphikmawan/ta2019:newpupp
	
	$ docker push cphikmawan/ta2019:newpupp
				\end{lstlisting}
				
		\subsection{Implementasi Pembuatan Lingkungan Kontainer}
			Kontainer akan dipasangkan pada suatu lingkungan yang bisa mengatur segala aktivitas kontainer, lingkungan yang akan dibangun pada sistem menggunakan alat orkestrasi yaitu Docker Swarm. Untuk mengimplementasikan Docker Swarm pada sistem ini, dibutuhkan satu node host sebagai manager node dan dua lainnya sebagai worker. Tahap pertama yang dilakukan yaitu menginisiasi salah satu node host yang akan digunakan sebagai manager node. Perintah inisiasi manager node terdapat pada Kode Sumber \ref{swarminit}. \\
			\begin{lstlisting}[frame=single,tabsize=2,breaklines,caption={Perintah untuk inisiasi manager node},label=swarminit, captionpos=b, language=json,numbers=none]
	$ docker swarm init --advertise-addr [IP NODE]
			\end{lstlisting}
			
			Setelah perintah pada kode sumber \ref{swarminit} dijalankan, maka manager node akan menghasilkan sebuah token yang digunakan oleh node host yang lain untuk bergabung sebagai worker. Perintah yang harus dijalankan pada setiap node host yang lain terdapat pada Kode Sumber \ref{swarmjoin}.
			\begin{lstlisting}[frame=single,tabsize=2,breaklines,caption={Perintah untuk bergabung ke Swarm},label=swarmjoin, captionpos=b, language=json,numbers=none]
	$ docker swarm join --token [token] [IP MANAGER]:2377
			\end{lstlisting}
			
			Tahap terakhir yang dilakukan yaitu memastikan semua node host sudah tergabung dengan manager node.
			\begin{lstlisting}[frame=single,tabsize=2,breaklines,caption={Perintah untuk melihat daftar Swarm Node},label=dockernodels, captionpos=b, language=json,numbers=none]
	$ docker node ls
			\end{lstlisting}
			
		\subsection{Implementasi Pemasangan Headless Chrome dan Puppeteer}
			Headless Chrome dan Puppeteer akan dipasang pada masing-masing kontainer menggunakan Docker Image yang telah dibuat sebelumnya, untuk pemasangannya akan dilakukan dilingkungan Docker Swarm dan dilakukan pada manager node. Namun untuk implementasinya dibutuhkan beberapa persiapan dan konfigurasi yang harus dilakukan terlebih dahulu yaitu konfigurasi unduh Docker Image, konfigurasi membuat Docker Network, konfigurasi puppeteer.yml, konfigurasi deployment.
			
			\indent untuk mengunduh Docker Image menggunakan perintah pada Kode Sumber \ref{unduhimage} dan membuat Docker Network pada Kode Sumber \ref{createnetwork}. Sedangkan konfigurasi puppeteer.yml dapat dilihat di Kode Sumber \ref{puppyaml}
			\begin{lstlisting}[frame=single,tabsize=2,breaklines,caption={Perintah untuk mengunduh Docker Image },label=unduhimage, captionpos=b, language=json,numbers=none]
	# dijalankan pada setiap node host
	$ docker image pull cphikmawan/ta2019:puppeteer
			\end{lstlisting}
			
			\begin{lstlisting}[frame=single,tabsize=2,breaklines,caption={Perintah untuk membuat Docker Network },label=createnetwork, captionpos=b, language=json,numbers=none]
	$ docker network create \
		--driver overlay \
		--subnet 10.0.0.0/18 \
		--attachable \
		[nama_network]
			\end{lstlisting}
			
			\begin{lstlisting}[frame=single,tabsize=2,breaklines,caption={Konfigurasi puppeteer.yml},label=puppyaml, captionpos=b, language=json]
	version: '3'
	# konfigurasi service
	services:
		# nama service yang akan dibuat
		puppeteer:
			# docker image yang digunakan
			image: cphikmawan/ta2019:newpupp
			# sinkronisasi penyimpanan antara kontainer dengan host
			volumes:
				- ./output:/app/output
				- ./code:/app/code
			# direktori kerja didalam kontainer
			working_dir: /app/code
			# konfigurasi untuk jumlah kontainer dan handling
			deploy:
				replicas: 1000
				restart_policy:
					condition: on-failure
			# entrypoint awal tidak akan melakukan apapun
			entrypoint: ["sh", "-c", "sleep infinity"]
	# konfigurasi jaringan
	networks:
		default:
			# jaringan default akan diubah ke jaringan eksternal
			external:
				# akan terkonek ke jaringan "swarm-network"
				name: swarm-network
			\end{lstlisting}
			
			\indent Jika persiapan diatas sudah dilakukan, maka tahapan selanjutnya adalah konfigurasi deployment dengan menjalankan perintah yang ditunjukkan pada Kode Sumber \ref{pemasanganstack} di terminal manager node
			\begin{lstlisting}[frame=single,tabsize=2,breaklines,caption={Perintah untuk pemasangan kontainer },label=pemasanganstack, captionpos=b, language=json,numbers=none]
	$ docker stack deploy --compose-file=puppeteer.yml [nama_stack]
			\end{lstlisting}
	
	\section{Implementasi Pengambil Data Uji Beban}
		Pengambil Data Uji Beban akan dilakukan menggunakan sebuah pustaka Node yaitu Puppeteer. Pada saat dilakukan uji beban, Puppeteer akan mengambil data uji beban dari Headless Chrome menggunakan API dari Puppeteer secara otomatis. Beberapa data uji beban yang akan diambil yaitu:
		\begin{enumerate}
			\item Response End \\
				Atribut ini menunjukkan waktu setelah user menerima byte terakhir dari dokumen sebelum koneksi transportasi ditutup.
			\item DOM Content Loaded \\
				Atribut ini menunjukkan waktu setelah dokumen sudah diterima oleh user.
			\item Load Event End \\
				Atribut ini mengembalikan waktu ketika memuat dokumen selesai.
			\item CSS Tracing End \\
				Atribut ini menunjukkan waktu dari ekstraksi akhir file CSS dimuat.
			\item First Meaningfulpain \\
				Atribut ini adalah atribut khusus yang ada pada Chrome yang menunjukkan bahwa segala konten halaman yang dimuat sudah ditampilkan di layar. \\
		\end{enumerate}
	 
	 	\indent Selain data uji beban diatas, Puppeteer juga digunakan untuk mengambil sebuah tangkapan layar sesuai skenario yang dikirimkan oleh pengguna dan mengambil data saat terjadi kegagalan saat memuat assets yang tertulis pada console browser. Adapun pseudocode untuk melakukan pengambilan data uji beban dapat dilihat pada Kode Sumber \ref{pseudocodepupp}.
	 	
	 	\begin{lstlisting}[frame=single,tabsize=2,breaklines,caption={Pseudocode Puppeteer },label=pseudocodepupp, captionpos=b, language=json]
 	Variable Declaration
 	Data = Read Scenario File Configuration
 	
 	TestPage Function:
 		Call Helpers Function:
 			Get Navigation Start
 		Start Trace CSS Data
 		Trying to Accessing Website
 		End Trace CSS Data
 		Call Helpers Function:
 			Get Extracted Performance Data
 			Get Extracted CSS Tracing
 			Send Extracted Data
 	
 	Helpers Function:
 		Get Navigation Start
 			Send Navigation Start
 		Extracted Performance Data = Performance Data * 1000 - Navigation Start
 			Send Extracted Performance Data
 		Extracted CSS Tracing = CSS Tracing / 1000
 			Send Extracted CSS Tracing 
 	
 	Main Function:
 		Start Headless Browser
 		Get Error Console
 			Send Data to Database
 		Try:
 			Call TestPage Function(Data):
 				Get Data From TestPage -> Save Data Uji Beban
 			Get Page Screenshoot -> Save Screenshoot
 		Catch Error:
 			Get Error
 			Get Page Screenshoot -> Save Screenshoot
	 	\end{lstlisting}
	
	\section{Implementasi Service Controller}
		Berdasarkan desain dan perancangan, service controller terdiri dari komponen web service, basis data dan task queue. Semua komponen tersebut akan diimplementasikan pada satu komputer milik penulis yang akan digunakan untuk web service dan penggunaan task queue, serta satu buah server untuk basis data MySQL.
		
		\subsection{Implementasi Web Service}
			Pada implementasi web service dibutuhkan beberapa persiapan lingkungan yang perlu dilakukan, urutannya meliputi langkah-langkah berikut:
			\begin{enumerate}
				\item Instalasi PHP
				\item Instalasi Composer
				\item Instalasi Laravel versi 5.8
				\item Instalasi MySQL
			\end{enumerate}
		
			\begin{figure}[H]
				\centering
				\includegraphics[width=10cm,height=6cm]{Images/C-4/gambarweb.png}
				\caption{Tampilan web antarmuka pengguna}
				\label{gambarweb}
			\end{figure}
			
			\indent Web service akan menggunakan bahasa PHP dan kerangka kerja Laravel versi 5.8, sedangkan Composer berfungsi untuk memanajemen instalasi pustaka pada PHP dan untuk penyimpanan data yang digunakan pada sistem akan disimpan pada basis data MySQL. Web service berfungsi untuk memudahkan user melakukan uji beban pada suatu web. Tampilan antarmuka pengguna ditunjukkan pada Gambar \ref{gambarweb}. \\
			
			\indent Web service memiliki rute-rute HTTP yang akan digunakan oleh user ketika mengakses web sistem. Rute-rute tersebut ditunjukkan pada Tabel \ref{tabelruteweb}.
			\begin{longtable}{|p{0.05\textwidth}|p{0.25\textwidth}|p{0.30\textwidth}|p{0.30\textwidth}|}
				\caption{Rute HTTP pada web service} \label{tabelruteweb} \\
				\hline
				\textbf{No} & \textbf{Rute} & \textbf{Metode} & \textbf{Aksi} \\ \hline
				\endhead
				\endfoot
				\endlastfoot
				1 & / & GET & Mengakses halaman home \\ \hline
				2 & /login & GET & Mengakses halaman login \\ \hline
				3 & /login & POST & Melakukan login \\ \hline
				4 & /logout & GET & Melakukan logout \\ \hline
				5 & /skenario & GET & Melihat skenario \\ \hline
				6 & /skenario & POST & Menambahkan skenario \\ \hline
				7 & /skenario & DELETE & Menghapus skenario \\ \hline
				8 & /worker & GET & Mengakses halaman worker \\ \hline
				9 & /worker & POST & Menambahkan jumlah worker dan data antrian \\ \hline
				10 & /hasil/rata-rata & GET & Mendapatkan hasil pengujian \\ \hline
				11 & /hasil/error-console & GET & Mendapatkan error console web \\ \hline
				12 & /hasil/images & GET & Melihat tangkapan layar web \\ \hline
				13 & /antrian & GET & Melihat jumlah dan status antrian \\ \hline
				
			\end{longtable}
		
		\subsection{Implementasi Skema Basis Data}
			Berdasarkan hasil perancangan basis data pada bab sebelumnya. Data yang dibutuhkan dan digunakan oleh sistem akan disimpan di dalam basis data MySQL. Data yang disimpan adalah data node host swarm, data kontainer, data pengguna, data skenario pengujian, data antrian request, data hasil pengujian, data error console, data rata-rata hasil pengujian.
			
			\subsubsection{Tabel Swarms}
			\begin{longtable}{|p{0.05\textwidth}|p{0.25\textwidth}|p{0.30\textwidth}|p{0.30\textwidth}|}
				\caption{Tabel swarms} \label{tabelruteweb} \\
				\hline
				\textbf{No} & \textbf{Kolom} & \textbf{Tipe} & \textbf{Keterangan} \\ \hline
				\endhead
				\endfoot
				\endlastfoot
				1 &  &  &  \\ \hline
			\end{longtable}
		
			\subsubsection{Tabel Containers}
			\begin{longtable}{|p{0.05\textwidth}|p{0.25\textwidth}|p{0.30\textwidth}|p{0.30\textwidth}|}
				\caption{Tabel containers} \label{tabelruteweb} \\
				\hline
				\textbf{No} & \textbf{Kolom} & \textbf{Tipe} & \textbf{Keterangan} \\ \hline
				\endhead
				\endfoot
				\endlastfoot
				1 &  &  &  \\ \hline
			\end{longtable}
		
			\subsubsection{Tabel Users}
			\begin{longtable}{|p{0.05\textwidth}|p{0.25\textwidth}|p{0.30\textwidth}|p{0.30\textwidth}|}
				\caption{Tabel users} \label{tabelruteweb} \\
				\hline
				\textbf{No} & \textbf{Kolom} & \textbf{Tipe} & \textbf{Keterangan} \\ \hline
				\endhead
				\endfoot
				\endlastfoot
				1 &  &  &  \\ \hline
			\end{longtable}
		
			\subsubsection{Tabel Scenarios}
			\begin{longtable}{|p{0.05\textwidth}|p{0.25\textwidth}|p{0.30\textwidth}|p{0.30\textwidth}|}
				\caption{Tabel scenarios} \label{tabelruteweb} \\
				\hline
				\textbf{No} & \textbf{Kolom} & \textbf{Tipe} & \textbf{Keterangan} \\ \hline
				\endhead
				\endfoot
				\endlastfoot
				1 &  &  &  \\ \hline
			\end{longtable}
		
			\subsubsection{Tabel Queues}
			\begin{longtable}{|p{0.05\textwidth}|p{0.25\textwidth}|p{0.30\textwidth}|p{0.30\textwidth}|}
				\caption{Tabel queues} \label{tabelruteweb} \\
				\hline
				\textbf{No} & \textbf{Kolom} & \textbf{Tipe} & \textbf{Keterangan} \\ \hline
				\endhead
				\endfoot
				\endlastfoot
				1 &  &  &  \\ \hline
			\end{longtable}
		
			\subsubsection{Tabel Results}
			\begin{longtable}{|p{0.05\textwidth}|p{0.25\textwidth}|p{0.30\textwidth}|p{0.30\textwidth}|}
				\caption{Tabel results} \label{tabelruteweb} \\
				\hline
				\textbf{No} & \textbf{Kolom} & \textbf{Tipe} & \textbf{Keterangan} \\ \hline
				\endhead
				\endfoot
				\endlastfoot
				1 &  &  &  \\ \hline
			\end{longtable}
		
			\subsubsection{Tabel Errors}
			\begin{longtable}{|p{0.05\textwidth}|p{0.25\textwidth}|p{0.30\textwidth}|p{0.30\textwidth}|}
				\caption{Tabel errors} \label{tabelruteweb} \\
				\hline
				\textbf{No} & \textbf{Kolom} & \textbf{Tipe} & \textbf{Keterangan} \\ \hline
				\endhead
				\endfoot
				\endlastfoot
				1 &  &  &  \\ \hline
			\end{longtable}
		
			\subsubsection{Tabel Summary Results}
			\begin{longtable}{|p{0.05\textwidth}|p{0.25\textwidth}|p{0.30\textwidth}|p{0.30\textwidth}|}
				\caption{Tabel summary results} \label{tabelruteweb} \\
				\hline
				\textbf{No} & \textbf{Kolom} & \textbf{Tipe} & \textbf{Keterangan} \\ \hline
				\endhead
				\endfoot
				\endlastfoot
				1 &  &  &  \\ \hline
			\end{longtable}
			
		\subsection{Implementasi Task Queue}
			Task queue yang akan digunakan adalah.
	
	\chapter{PENGUJIAN DAN EVALUASI}
	Pada bab ini akan dibahas uji coba dan evaluasi dari sistem yang dibuat. Sistem akan diuji coba fungsionalitasnya dengan menjalankan skenario pengujian performa pada web. Uji coba dilakukan untuk mengetahui kinerja sistem dengan lingkungan uji coba yang ditentukan.
	
	\section{Lingkungan Uji Coba}
		Lingkungan Uji coba sistem ini terdiri dari beberapa komponen yaitu web service, server basis data, server manager node, dua server worker. Server yang digunakan sistem menggunakan layanan Virtual Priate Server dari DigitalOcean, sedangkan web service akan dibangun di komputer penulis. Spesifikasi untuk setiap komponen ditunjukkan pada Tabel \ref{tabelspesifikasi}.
		\begin{longtable}{|p{0.05\textwidth}|p{0.25\textwidth}|p{0.30\textwidth}|p{0.30\textwidth}|}
			\caption{Spesifikasi komponen} \label{tabelspesifikasi} \\
			\hline
			\textbf{No} & \textbf{Komponen} & \textbf{Perangkat Keras} & \textbf{Perangkat Lunak} \\ \hline
			\endhead
			\endfoot
			\endlastfoot
			1 & Web Service \& Task Queue & Processor AMD FX-7600P Radeon R7, 4 Core, 8GB RAM, 250GB SSD & Ubuntu 18.04 LTS, Laravel 5.8, Python 3.6 \\ \hline
			2 & Basis Data & 1 Core Processor, 1GB RAM, 20GB SSD & Ubuntu 18.04 LTS, MySQL 5.7 \\ \hline
			3 & Manager Node & 2 Core Processor, 4GB RAM, 80GB SSD & Ubuntu 18.04 LTS, Python 3.6, Docker 18.09.6, Node.js 8.15, NPM 6.4.1, Chrome, Puppeteer 0.12.0, MySQL Client 5.7 \\ \hline
			4 & Worker 1 & 2 Core Processor, 4GB RAM, 80GB SSD & Ubuntu 18.04 LTS, Python 3.6, Docker 18.09.6, Node.js 8.15, NPM 6.4.1, Chrome, Puppeteer 0.12.0, MySQL Client 5.7 \\ \hline
			5 & Worker 2 & 2 Core Processor, 4GB RAM, 80GB SSD & Ubuntu 18.04 LTS, Python 3.6, Docker 18.09.6, Node.js 8.15, NPM 6.4.1, Chrome, Puppeteer 0.12.0, MySQL Client 5.7 \\ \hline
		\end{longtable}
	
		\indent Untuk akses ke setiap komponen, digunakan IP publik yang disediakan untuk masing-masing komponen. Detail ditunjukkan pada Tabel \ref{ipserver}.
		\begin{longtable}{|p{0.05\textwidth}|p{0.25\textwidth}|p{0.30\textwidth}|p{0.30\textwidth}|}
			\caption{IP dan hostname server} \label{ipserver} \\
			\hline
			\textbf{No} & \textbf{Komponen} & \textbf{IP} & \textbf{Hostname} \\ \hline
			\endhead
			\endfoot
			\endlastfoot
			1 & Web Service & 10.151.253.110 & night \\ \hline
			2 & Basis Data & 178.128.123.143 & NIGHT \\ \hline
			3 & Manager Node & 167.71.194.235 & CLOUD \\ \hline
			4 & Worker Node 1 & 165.22.55.82 & RAIN \\ \hline
			5 & Worker Node 2 & 167.71.194.233 & STORM \\ \hline
		\end{longtable}
	
	\section{Skenario Uji Coba} \label{skenarioujicoba}
		Uji coba ini dilakukan untuk menguji apakah fungsionalitas yang diidentifikasikan terhadap kebutuhan sistem benar-benar telah diimplementasikan dan bekerja seperti yang seharusnya. Pengujian yang dilakukan didasarkan pada fungsionalitas yang disajikan sistem.
		
	\subsection{Skenario Uji Fungsionalitas}
		Uji fungsionalitas dibagi menjadi beberapa bagian yaitu pembuatan load generator docker container, user mengelola skenario melalui web, user mengirim request uji beban melalui web, penggunaan task queue, pengambilan data uji beban dan user melihat hasil uji beban melalui web.
		
		\subsubsection{Uji Fungsionalitas Pembuatan Load Generator Docker Container}
		
		\subsubsection{Uji Fungsionalitas User Mengelola Skenario Melalui Web}
			Dilakukan pengelolaan skenario oleh user pada waktu me
		
		\subsubsection{Uji Fungsionalitas User Mengirim Request Uji Beban Melalui Web}
		
		\subsubsection{Uji Fungsionalitas Penggunaan Task Queue Terhadap Request}
		
		\subsubsection{Uji Fungsionalitas Pengambil Data Uji Beban}
		
		\subsubsection{Uji Fungsionalitas User Melihat Hasil Uji Beban Melalui Web}
	
	\section{Hasil Uji Coba dan Evaluasi}
		Berikut dijelaskan hasil uji coba dan evaluasi berdasarkan skenario yang sudah dijelaskan pada bab \ref{skenarioujicoba}.
		
	\subsection{Uji Fungsionalitas}
		Berikut dijelaskan hasil pengujian fungsionalitas pada sistem yang sudah dibangun.
	
		\subsubsection{Uji Fungsionalitas Pembuatan Load Generator Docker Container}
		
		\subsubsection{Uji Fungsionalitas User Mengelola Skenario Melalui Web}
		
		\subsubsection{Uji Fungsionalitas User Mengirim Request Uji Beban Melalui Web}
		
		\subsubsection{Uji Fungsionalitas Penggunaan Task Queue Terhadap Request}
		
		\subsubsection{Uji Fungsionalitas Pengambil Data Uji Beban}
		
		\subsubsection{Uji Fungsionalitas User Melihat Hasil Uji Beban Melalui Web}
	\chapter{PENUTUP}
    Bab ini membahas kesimpulan yang dapat diambil dari tujuan pembuatan sistem dan hubungannya dengan hasil uji coba dan evaluasi yang telah dilakukan. Selain itu, terdapat beberapa saran yang bisa dijadikan acuan untuk melakukan pengembangan dan penelitian lebih lanjut.
        
	\section{Kesimpulan}
        Dari proses perencangan, implementasi dan pengujian terhadap sistem, dapat diambil beberapa kesimpulan berikut:
        \begin{enumerate}
        	\item Dalam pengembangan web, uji beban dapat dilakukan menggunakan \textit{Headless Chrome} sebagai \textit{tester}, yang dipadukan dengan \textit{API} pengambil uji beban yaitu \textit{Puppeteer}.
     		\item Load generator yang digunakan sebagai resource user dapat diimplementasikan oleh Docker. Docker dipasang menggunakan Docker Image yang sudah diinstalasi \textit{Node.js} dan \textit{Puppeteer} didalamnya.
        	\item Data Laporan uji beban didapatkan menggunakan \textit{Puppeteer} dan disajikan pada web dalam bentuk tabel dengan data inti yaitu \textit{Response End}, \textit{CSS Tracing End}, \textit{Dom Content Loaded}, \textit{First Meaningful Paint} dan \textit{Load Event End}. Laporan yang disajikan juga menampilkan \textit{error console} yang terekam pada \textit{browser} dan tangkapan layar web yang diuji.
        	\item Dalam menangani layanan uji beban yang dikirimkan oleh pengembang web, sistem menggunakan sebuah \textit{task scheduler crontab} yang menjalankan \textit{script Python} setiap satu menit dan melakukan pengecekan secara berkala.
        \end{enumerate}
        
	\section{Saran}
		Berikut beberapa saran yang diberikan untuk pengembangan lebih lanjut:
		\begin{itemize}
			\item Sistem sudah bisa membuat sebuah \textit{tester} dan \textit{load generator}, namun masih memiliki kendala terhadap penggunaan \textit{CPU} dan \textit{RAM} ketika menjalankan \textit{Headless Chrome} secara bersama, oleh karena itu pengembang bisa menambahkan sistem untuk melakukan \textit{multithreading} ketika \textit{load generator} dijalankan bersama, untuk saat ini sistem hanya membatasi penggunaan \textit{load generator}.
			\item \textit{Headless Chrome} dan \textit{Puppeteer} tergolong tools yang baru, namun perkembangannya sangat cepat. Untuk kedepannya pengembang bisa memadukan \textit{Headless Chrome} dan \textit{Puppeteer} dengan \textit{tools load test} yang lain misalnya \textit{K6}, \textit{Jmeter Rest API}, \textit{Selenium} atau \textit{Firefox Headless Mode}.
			\item Untuk menangani permintaan jumlah \textit{load generator} yang tinggi, server yang digunakan harus diperbanyak agar dapat menunjang sistem untuk melayani uji beban.
			\item Sistem dapat melakukan uji beban dengan metode \textit{GET} dan dapat menghasilkan beberapa laporan uji beban, kedepannya bisa dikembangkan untuk uji beban menggunakan metode \textit{POST}. Serta pengembang juga bisa mempercantik tampilan antarmuka pengguna agar lebih nyaman untuk digunakan.
		\end{itemize}

	\bibliography{Zotero}
	\bibliographystyle{IEEEtranID.bst}
    
    \renewcommand\chaptername{LAMPIRAN}
	\appendix
    \chapter{INSTALASI PERANGKAT LUNAK}

\section*{Instalasi \textit{Docker}}
	Untuk melakukan instalasi \textit{Docker}, dilakukan seperti langkah-langkah berikut:
\begin{lstlisting}[frame=single,tabsize=2,breaklines,caption={Perintah instalasi Docker },label=instalasidocker, captionpos=b, language=json,numbers=none]
$ sudo apt-get -y install \
apt-transport-https \
ca-certificates \
curl

$ curl -fsSL https://download.docker.com/linux/
ubuntu/gpg | sudo apt-key add -

$ sudo add-apt-repository \
"deb [arch=amd64] https://download.docker.com/linux/ubuntu \
$(lsb_release -cs) \
stable"

$ sudo apt-get update

$ sudo apt-get install docker-ce docker-ce-cli containerd.io
\end{lstlisting}
	
	Setelah menjalankan perintah pada kode sumber \ref{instalasidocker}. Jalankan perintah berikut agar \textit{Docker} bisa dijalankan sebagai \textit{Non-Root User}.
\begin{lstlisting}[frame=single,tabsize=2,breaklines,caption={Perintah mengubah hak User },label=nonrootuser, captionpos=b, language=json,numbers=none]
$ sudo groupadd docker
$ sudo usermod -aG docker $USER
\end{lstlisting}

\section*{Instalasi \textit{Docker Compose}}
\textit{Docker Compose} digunakan untuk otomasi dalam menjalankan \textit{Dockerfile} dan berperan untuk pembuatan Docker Image. Instalasi \textit{Docker Compose} dapat dilihat pada kode sumber \ref{dockercomposeinstall}.

\begin{lstlisting}[frame=single,tabsize=2,breaklines,caption={Perintah instalasi Docker Compose},label=dockercomposeinstall, captionpos=b, language=json,numbers=none]
$ sudo curl -L "https://github.com/docker/compose/releases/download/1.24.0/docker-compose-$(uname -s)-$(uname -m)" -o /usr/local/bin/docker-compose

$ sudo chmod +x /usr/local/bin/docker-compose

$ sudo ln -s /usr/local/bin/docker-compose /usr/bin/docker-compose
\end{lstlisting}

\section*{Instalasi \textit{Linux Package}}
Beberapa \textit{package} yang dibutuhkan dalam pembuatan sistem.
\begin{itemize}
	\item Node.js \\
		\texttt{\$ sudo apt install nodejs}\\
	\item NPM \\
		\texttt{\$ sudo apt install npm}\\
	\item MySQL \\
		\texttt{\$ sudo apt install mysql-server mysql-client}\\
	\item PIP \\
		\texttt{\$ sudo apt install python3-pip}\\
	\item mysql-connector-python \\
		\texttt{\$ pip3 install mysql-connector-python}\\
	\item Composer \\
		\texttt{\$ php -r "copy('https://getcomposer.org/installer', 'composer-setup.php');"} \\ \\
		\texttt{\$ php -r "if (hash\_file('sha384', 'composer-setup.php') === '48e3236262b34d30969dca3c37281b3b4bbe3221bda826ac6a9a62d6444cdb0dcd0615698a5cbe587c3f0fe57a54d8f5') { echo 'Installer verified'; } else { echo 'Installer corrupt'; unlink('composer-setup.php'); } echo PHP\_EOL;"}\\ \\
		\texttt{\$ php composer-setup.php} \\
		\texttt{\$ php -r "unlink('composer-setup.php');"} \\
		\texttt{\$ mv composer.phar /usr/local/bin/composer} \\
	\item Laravel \\
		\texttt{\$ composer global require laravel/installer}\\
\end{itemize}
    \chapter{KODE SUMBER}

\section*{Kode Sumber Pengambilan Data \textit{Metrics Performance}} \label{puppeteer}
	\subsection*{Isi berkas index.js}
\begin{lstlisting}[frame=single,tabsize=2,breaklines,caption={Isi berkas index.js},label=indexjs, captionpos=b, language=json]
const puppeteer = require('puppeteer');
const testPage = require('./testPage');
const fs = require('fs');
const db = require('../config/databases');
const scenario_id = process.argv[2];
const counter = process.argv[3];
const worker = process.argv[4];
const host = process.argv[5];

let rawdata = fs.readFileSync('/app/code/assets/config_' + scenario_id + '.json');
let config = JSON.parse(rawdata);
(async () => {
	const browser = await puppeteer.launch({ args: ['--no-sandbox'] });
	const page = await browser.newPage();
	await page.on('console', msg =>
		db.query('INSERT INTO errors (scenario_id, link, worker, username, host, type, text, location_url) VALUES (?, ?, ?, ?, ?, ?, ?, ?)',
		[scenario_id, config.scenario_link, worker, config.username, host, msg._type, msg._text, msg._location.url])
	);
	try {
		let data = await testPage(page, config, counter);
		db.query('INSERT INTO results (scenario_id, link, method, worker, username, host, response_end, dom_content_load, load_event_end, css_trace_end, first_meaningful) VALUES (?, ?, ?, ?, ?, ?, ?, ?, ?, ?, ?)',
			[scenario_id, config.scenario_link, config.scenario_method, worker, config.username, host, data.Timing.responseEnd, data.Timing.domContentLoadedEventEnd, data.Timing.loadEventEnd, data.TraceResult.cssEnd, data.Metrics.FirstMeaningfulPaint]);
		db.end();
		await page.screenshot({ path: '/app/output/ss' + scenario_id + '.png' });
		await browser.close();
	} catch (error) {
		console.error(error);
		db.query('INSERT INTO results (scenario_id, link, method, worker, username, host, response_end, dom_content_load, load_event_end, css_trace_end, first_meaningful) VALUES (?, ?, ?, ?, ?, ?, ?, ?, ?, ?, ?)',
			[scenario_id, config.scenario_link, config.scenario_method, worker, config.username, host, -1, -1, -1, -1, -1]);
		db.end();
		await page.screenshot({ path: '/app/output/ss' + scenario_id + '.png' });
		await browser.close();
	}
})();
\end{lstlisting}

	\subsection*{Isi berkas testPage.js}
\begin{lstlisting}[frame=single,tabsize=2,breaklines,caption={Isi berkas testPage.js},label=testjs, captionpos=b, language=json]
const {
	getTimeFromPerformanceMetrics,
	extractDataFromPerformanceMetrics,
	extractDataFromPerformanceTiming,
	extractDataFromTracing,
} = require('./helpers');

async function testPage(page, config, counter) {
	const client = await page.target().createCDPSession();
	await client.send('Performance.enable');
	const navigationStart = getTimeFromPerformanceMetrics(
		await client.send('Performance.getMetrics'),
		'NavigationStart'
	);
	
	await page.tracing.start({ path: './trace' + config.scenario_id + counter + '.json' });
	
	await page.goto(config.scenario_link);
	
	const performanceTiming = JSON.parse(
		await page.evaluate(() => JSON.stringify(window.performance.timing))
	);
	
	let firstMeaningfulPaint = 0;
	while (firstMeaningfulPaint === 0) {
		await page.waitFor(300);
		performanceMetrics = await client.send('Performance.getMetrics');
		firstMeaningfulPaint = getTimeFromPerformanceMetrics(
			performanceMetrics, 'FirstMeaningfulPaint'
		);
	}
	
	await page.tracing.stop();
	
	const cssTracing = await extractDataFromTracing(
		'./trace' + config.scenario_id + counter + '.json',
		config.scenario_link,
	);
	
	let TraceResult = {
		cssEnd: cssTracing.end - navigationStart,
	}
	
	let Metrics = extractDataFromPerformanceMetrics(
		performanceMetrics,
		'FirstMeaningfulPaint',
	);
	
	let Timing = extractDataFromPerformanceTiming(
	performanceTiming,
		'responseEnd',
		'domContentLoadedEventEnd',
		'loadEventEnd',
	);
	
	return { TraceResult, Metrics, Timing };
}

module.exports = testPage;
\end{lstlisting}

	\subsection*{Isi berkas helpers.js}
\begin{lstlisting}[frame=single,tabsize=2,breaklines,caption={Isi berkas helpers.js},label=helperjs, captionpos=b, language=json]
const fs = require('fs');

const getTimeFromPerformanceMetrics = (metrics, name) =>
	metrics.metrics.find(x => x.name === name).value * 1000;

const extractDataFromPerformanceMetrics = (metrics, ...dataNames) => {
	const navigationStart = getTimeFromPerformanceMetrics(
		metrics,
		'NavigationStart'
	);
	const extractedData = {};
	dataNames.forEach(name => {
		extractedData[name] =
			getTimeFromPerformanceMetrics(metrics, name) - navigationStart;
	});
	return extractedData;
};

const extractDataFromPerformanceTiming = (timing, ...dataNames) => {
	const navStart = timing.navigationStart;
	
	const extractedData = {};
	dataNames.forEach(name => {
		extractedData[name] = timing[name] - navStart;
	});
	
	return extractedData;
};

const extractDataFromTracing = (path, link) => new Promise(resolve => {
	const tracing = JSON.parse(fs.readFileSync(path, 'utf8'));
	const resourceTracings = tracing.traceEvents.filter(
		x =>
			x.cat === 'devtools.timeline' &&
			typeof x.args.data !== 'undefined' &&
			typeof x.args.data.url !== 'undefined' &&
			x.args.data.url.includes(link)
	);
	const resourceTracingSendRequest = resourceTracings.find(
		x => x.name === 'ResourceSendRequest'
	);
	const resourceId = resourceTracingSendRequest.args.data.requestId;
	const resourceTracingEnd = tracing.traceEvents.filter(
		x =>
			x.cat === 'devtools.timeline' &&
			typeof x.args.data !== 'undefined' &&
			typeof x.args.data.requestId !== 'undefined' &&
			x.args.data.requestId === resourceId
	);
	const resourceTracingStartTime = resourceTracingSendRequest.ts / 1000;
	const resourceTracingEndTime =
		resourceTracingEnd.find(x => x.name === 'ResourceFinish').ts / 1000;
	
	fs.unlink(path, () => {
		resolve({
			end: resourceTracingEndTime,
		});
	});
});

module.exports = {
	getTimeFromPerformanceMetrics,
	extractDataFromPerformanceMetrics,
	extractDataFromPerformanceTiming,
	extractDataFromTracing,
};
\end{lstlisting}

\section*{Kode Sumber Basis Data \textit{MySQL}} \label{mysql}
	
\begin{lstlisting}[frame=single,tabsize=2,breaklines,caption={Basis data MySQL},label=mysql, captionpos=b, language=json]
CREATE TABLE `swarms` (
	`id` bigint(20) unsigned NOT NULL AUTO_INCREMENT,
	`created_at` timestamp NULL DEFAULT NULL,
	`updated_at` timestamp NULL DEFAULT NULL,
	`swarm_ip` varchar(255) COLLATE utf8mb4_unicode_ci NOT NULL,
	`swarm_username` varchar(255) COLLATE utf8mb4_unicode_ci NOT NULL,
	`swarm_password` varchar(255) COLLATE utf8mb4_unicode_ci NOT NULL,
	`is_used` varchar(255) COLLATE utf8mb4_unicode_ci NOT NULL,
	PRIMARY KEY (`id`),
	UNIQUE KEY `swarms_swarm_ip_unique` (`swarm_ip`)
) ENGINE=InnoDB AUTO_INCREMENT=5 DEFAULT CHARSET=utf8mb4 COLLATE=utf8mb4_unicode_ci;

CREATE TABLE `containers` (
	`id` bigint(20) unsigned NOT NULL AUTO_INCREMENT,
	`task_id` varchar(255) COLLATE utf8mb4_unicode_ci NOT NULL,
	`node_id` varchar(255) COLLATE utf8mb4_unicode_ci NOT NULL,
	`container_id` varchar(255) COLLATE utf8mb4_unicode_ci NOT NULL,
	`node_ip` varchar(255) COLLATE utf8mb4_unicode_ci NOT NULL,
	`node_host` varchar(255) COLLATE utf8mb4_unicode_ci NOT NULL,
	`status` smallint(6) NOT NULL DEFAULT '0',
	`username` varchar(100) COLLATE utf8mb4_unicode_ci NOT NULL DEFAULT '0',
	PRIMARY KEY (`id`)
) ENGINE=InnoDB AUTO_INCREMENT=101 DEFAULT CHARSET=utf8mb4 COLLATE=utf8mb4_unicode_ci;

CREATE TABLE `users` (
	`id` bigint(20) unsigned NOT NULL AUTO_INCREMENT,
	`name` varchar(255) COLLATE utf8mb4_unicode_ci NOT NULL,
	`email` varchar(255) COLLATE utf8mb4_unicode_ci NOT NULL,
	`username` varchar(255) COLLATE utf8mb4_unicode_ci NOT NULL,
	`email_verified_at` timestamp NULL DEFAULT NULL,
	`password` varchar(255) COLLATE utf8mb4_unicode_ci NOT NULL,
	`remember_token` varchar(100) COLLATE utf8mb4_unicode_ci DEFAULT NULL,
	`created_at` timestamp NULL DEFAULT NULL,
	`updated_at` timestamp NULL DEFAULT NULL,
	`request_test` varchar(100) COLLATE utf8mb4_unicode_ci DEFAULT NULL,
	PRIMARY KEY (`id`),
	UNIQUE KEY `users_email_unique` (`email`),
	UNIQUE KEY `users_username_unique` (`username`)
) ENGINE=InnoDB AUTO_INCREMENT=6 DEFAULT CHARSET=utf8mb4 COLLATE=utf8mb4_unicode_ci;

CREATE TABLE `role_user` (
	`id` bigint(20) unsigned NOT NULL AUTO_INCREMENT,
	`created_at` timestamp NULL DEFAULT NULL,
	`updated_at` timestamp NULL DEFAULT NULL,
	`role_id` int(10) unsigned NOT NULL,
	`user_id` int(10) unsigned NOT NULL,
	PRIMARY KEY (`id`)
) ENGINE=InnoDB AUTO_INCREMENT=5 DEFAULT CHARSET=utf8mb4 COLLATE=utf8mb4_unicode_ci;

CREATE TABLE `roles` (
	`id` bigint(20) unsigned NOT NULL AUTO_INCREMENT,
	`created_at` timestamp NULL DEFAULT NULL,
	`updated_at` timestamp NULL DEFAULT NULL,
	`name` varchar(255) COLLATE utf8mb4_unicode_ci NOT NULL,
	`description` varchar(255) COLLATE utf8mb4_unicode_ci NOT NULL,
	PRIMARY KEY (`id`)
) ENGINE=InnoDB AUTO_INCREMENT=3 DEFAULT CHARSET=utf8mb4 COLLATE=utf8mb4_unicode_ci;

CREATE TABLE `scenarios` (
	`id` bigint(20) unsigned NOT NULL AUTO_INCREMENT,
	`created_at` timestamp NULL DEFAULT NULL,
	`updated_at` timestamp NULL DEFAULT NULL,
	`scenario_id` varchar(255) COLLATE utf8mb4_unicode_ci NOT NULL,
	`username` varchar(255) COLLATE utf8mb4_unicode_ci NOT NULL,
	`scenario_method` varchar(255) COLLATE utf8mb4_unicode_ci NOT NULL,
	`scenario_link` varchar(255) COLLATE utf8mb4_unicode_ci NOT NULL,
	`scenario_worker` varchar(255) COLLATE utf8mb4_unicode_ci DEFAULT NULL,
	`scenario_status` smallint(6) NOT NULL DEFAULT '0',
	`scenario_button` varchar(255) COLLATE utf8mb4_unicode_ci DEFAULT NULL,
	PRIMARY KEY (`id`)
) ENGINE=InnoDB AUTO_INCREMENT=4 DEFAULT CHARSET=utf8mb4 COLLATE=utf8mb4_unicode_ci;

CREATE TABLE `queues` (
	`id` bigint(20) unsigned NOT NULL AUTO_INCREMENT,
	`created_at` timestamp NULL DEFAULT NULL,
	`updated_at` timestamp NULL DEFAULT NULL,
	`username` varchar(255) COLLATE utf8mb4_unicode_ci DEFAULT NULL,
	`worker` int(11) DEFAULT NULL,
	`status` smallint(6) DEFAULT NULL,
	PRIMARY KEY (`id`)
) ENGINE=InnoDB AUTO_INCREMENT=3 DEFAULT CHARSET=utf8mb4 COLLATE=utf8mb4_unicode_ci;

CREATE TABLE `results` (
	`id` bigint(20) unsigned NOT NULL AUTO_INCREMENT,
	`scenario_id` varchar(255) COLLATE utf8mb4_unicode_ci DEFAULT NULL,
	`link` varchar(255) COLLATE utf8mb4_unicode_ci DEFAULT NULL,
	`method` varchar(100) COLLATE utf8mb4_unicode_ci DEFAULT NULL,
	`worker` varchar(255) COLLATE utf8mb4_unicode_ci DEFAULT NULL,
	`username` varchar(100) COLLATE utf8mb4_unicode_ci DEFAULT NULL,
	`host` varchar(100) COLLATE utf8mb4_unicode_ci DEFAULT NULL,
	`response_end` varchar(100) COLLATE utf8mb4_unicode_ci DEFAULT NULL,
	`dom_content_load` varchar(100) COLLATE utf8mb4_unicode_ci DEFAULT NULL,
	`load_event_end` varchar(100) COLLATE utf8mb4_unicode_ci DEFAULT NULL,
	`css_trace_end` varchar(100) COLLATE utf8mb4_unicode_ci DEFAULT NULL,
	`first_meaningful` varchar(100) COLLATE utf8mb4_unicode_ci DEFAULT NULL,
	`status` smallint(6) DEFAULT '0',
	PRIMARY KEY (`id`)
) ENGINE=InnoDB AUTO_INCREMENT=126 DEFAULT CHARSET=utf8mb4 COLLATE=utf8mb4_unicode_ci;

CREATE TABLE `errors` (
	`id` bigint(20) unsigned NOT NULL AUTO_INCREMENT,
	`scenario_id` varchar(255) COLLATE utf8mb4_unicode_ci DEFAULT NULL,
	`link` varchar(255) COLLATE utf8mb4_unicode_ci DEFAULT NULL,
	`worker` varchar(255) COLLATE utf8mb4_unicode_ci DEFAULT NULL,
	`username` varchar(100) COLLATE utf8mb4_unicode_ci DEFAULT NULL,
	`host` varchar(100) COLLATE utf8mb4_unicode_ci DEFAULT NULL,
	`type` varchar(100) COLLATE utf8mb4_unicode_ci DEFAULT NULL,
	`text` varchar(255) COLLATE utf8mb4_unicode_ci DEFAULT NULL,
	`args` varchar(255) COLLATE utf8mb4_unicode_ci DEFAULT NULL,
	`location_url` varchar(255) COLLATE utf8mb4_unicode_ci DEFAULT NULL,
	PRIMARY KEY (`id`)
) ENGINE=InnoDB AUTO_INCREMENT=126 DEFAULT CHARSET=utf8mb4 COLLATE=utf8mb4_unicode_ci;

CREATE TABLE `summary_results` (
	`id` bigint(20) unsigned NOT NULL AUTO_INCREMENT,
	`scenario_id` varchar(255) COLLATE utf8mb4_unicode_ci DEFAULT NULL,
	`link` varchar(255) COLLATE utf8mb4_unicode_ci DEFAULT NULL,
	`method` varchar(100) COLLATE utf8mb4_unicode_ci DEFAULT NULL,
	`worker` varchar(255) COLLATE utf8mb4_unicode_ci DEFAULT NULL,
	`username` varchar(100) COLLATE utf8mb4_unicode_ci DEFAULT NULL,
	`error` varchar(100) COLLATE utf8mb4_unicode_ci DEFAULT NULL,
	`response_end` varchar(100) COLLATE utf8mb4_unicode_ci DEFAULT NULL,
	`dom_content_load` varchar(100) COLLATE utf8mb4_unicode_ci DEFAULT NULL,
	`load_event_end` varchar(100) COLLATE utf8mb4_unicode_ci DEFAULT NULL,
	`css_trace_end` varchar(100) COLLATE utf8mb4_unicode_ci DEFAULT NULL,
	`first_meaningful` varchar(100) COLLATE utf8mb4_unicode_ci DEFAULT NULL,
	PRIMARY KEY (`id`)
) ENGINE=InnoDB AUTO_INCREMENT=3 DEFAULT CHARSET=utf8mb4 COLLATE=utf8mb4_unicode_ci;
\end{lstlisting}

\section*{Kode Sumber \textit{Task Queue}} \label{taskqueue}

\begin{lstlisting}[frame=single,tabsize=2,breaklines,caption={Isi berkas connection.py},label=connectionpy, captionpos=b, language=json]
import mysql.connector
from mysql.connector import Error

def connect():
	connection = mysql.connector.connect(
		host="178.128.123.143",
		user="cloudy",
		passwd="sembarang12",
		database="tugas_akhir_2019"
	)
	return connection
\end{lstlisting}

\begin{lstlisting}[frame=single,tabsize=2,breaklines,caption={Isi berkas queue.py},label=queuepy, captionpos=b, language=json]
from connection import connect
from sys import argv
import os
import subprocess
from pathlib import Path

def checkDb():
	con = connect()
	cursor = con.cursor()
	
	# get antrian request
	getQueueStatus = ('SELECT * FROM queues WHERE status < 2 ORDER BY created_at LIMIT 1')
	cursor.execute(getQueueStatus)
	result = cursor.fetchone()
	count = cursor.rowcount
	return count

def getQueue():
	# connection
	con = connect()
	cursor = con.cursor()
	
	# get antrian request
	getQueueStatus = ('SELECT * FROM queues WHERE status < 2 ORDER BY created_at LIMIT 1')
	cursor.execute(getQueueStatus)
	result = cursor.fetchone()
	id_queue = result[0]
	username = result[3]
	status = result[5]
	
	# update status antrian menjadi 1 atau proses
	if status==0:
		updateQueueStatus = ('UPDATE queues SET status=1 WHERE username=%s AND id=%s')
		cursor.execute(updateQueueStatus,(username, id_queue,))
		con.commit()

def runScenario():
	# connection
	con = connect()
	cursor = con.cursor()
	
	# get queue status = 1
	getQueueStatus = ('SELECT * FROM queues WHERE status = 1 ORDER BY created_at LIMIT 1')
	cursor.execute(getQueueStatus)
	result = cursor.fetchone()
	count = cursor.rowcount
	if count > 0:
		id_queue = result[0]
		username = result[3]
		worker = result[4]
		
		# update status kontainer
		updateContainerStatus = ('UPDATE containers SET status = 1, username = %s WHERE status = 0 LIMIT %s')
		cursor.execute(updateContainerStatus,(username,worker,))
		con.commit()
		
		# get info swarm untuk sshpass
		getSwarmConnection = ('SELECT DISTINCT(c.node_ip), s.swarm_username, s.swarm_password, c.username,COUNT(c.node_ip) cc \
			FROM containers c JOIN swarms s \
			WHERE s.swarm_ip = c.node_ip AND c.status = 1 AND c.username = %s \
			GROUP BY c.node_ip HAVING cc > 1')
		cursor.execute(getSwarmConnection,(username,))
		swarms = cursor.fetchall()
		
		# run sshpass untuk eksekusi skenario
		print('process...')
		path='/home/cloudy/tugas-akhir-2019/implementasi/scripts/run_scenario.py'
		for swarm in swarms:
			ip = swarm[0]
			user = swarm[1]
			passw = swarm[2]
			command_line = ('sshpass -p %s ssh -o UserKnownHostsFile=/dev/null -o StrictHostKeyChecking=no %s@%s' % (passw, user, ip))
			cmd = ('python3 %s %s %s' % (path, username, ip))
			run = command_line.split()
			run.append(cmd)
			subprocess.Popen(run, stdout=subprocess.PIPE, stderr=open(os.devnull, 'w'))
		
		# update status running
		updateQueueStatus = ('UPDATE queues SET status = -1 WHERE username=%s AND id=%s')
		cursor.execute(updateQueueStatus,(username, id_queue,))
		con.commit()

def calculateResult():
	# connection
	con = connect()
	cursor = con.cursor()
	
	# get queue request is done
	getQueueStatus = ('SELECT * FROM queues WHERE status = -1 ORDER BY created_at LIMIT 1')
	cursor.execute(getQueueStatus)
	queue = cursor.fetchone()
	username = queue[3]
	
	# get count result
	getCountResult = ('SELECT r.scenario_id, s.scenario_worker, count(r.scenario_id) AS jml \
	FROM results r JOIN scenarios s ON s.scenario_id = r.scenario_id \
	WHERE r.username=%s AND r.status = 0 GROUP BY r.scenario_id, s.scenario_worker')
	cursor.execute(getCountResult,(username,))
	result = cursor.fetchone()
	check = cursor.rowcount
	if check > 0:
		scena_id = result[0]
		worker = int(result[1])
		hasil = result[2]
		if worker == hasil:
			getAvgResult = ('SELECT t.scenario_id, s.scenario_link, s.scenario_method, s.scenario_worker ,t.username, \
				(s.scenario_worker - count(t.scenario_id))/s.scenario_worker*100 AS error, \
				AVG(t.response_end) AS response_end, \
				AVG(t.dom_content_load) AS dom_content_load, \
				AVG(t.load_event_end) AS load_event_end, \
				AVG(t.css_trace_end) AS css_trace_end, \
				AVG(t.first_meaningful) AS first_meaningful \
				FROM scenarios s INNER JOIN results t ON t.scenario_id = s.scenario_id \
				WHERE s.username = %s AND load_event_end > 0 AND s.scenario_id \
				NOT IN (SELECT scenario_id FROM summary_results) \
				GROUP BY scenario_id, s.scenario_worker, s.scenario_link, s.scenario_method, t.username')
			cursor.execute(getAvgResult,(username,))
			results = cursor.fetchone()
			check_res = cursor.rowcount
			if check > 0:
				scen_id = results[0]
				link = results[1]
				method = results[2]
				scen_worker = results[3]
				username = results[4]
				error = results[5]
				response = results[6]
				dom = results[7]
				load = results[8]
				css = results[9]
				fmfp = results[10]
				
				insertAvgResult = ('INSERT INTO summary_results (scenario_id, link, method, worker, username, error, response_end, dom_content_load, load_event_end, css_trace_end, first_meaningful) \
				values (%s, %s, %s, %s, %s, %s, %s, %s, %s, %s, %s)')
				cursor.execute(insertAvgResult,(scen_id, link, method, scen_worker, username, error, response, dom, load, css, fmfp,))
				con.commit()
				
				updateQueueStatus = ('UPDATE queues SET status = 2 WHERE status = -1')
				updateResultStatus = ('UPDATE results SET status = 1 WHERE scenario_id = %s')
				updateContainerStatus = ('UPDATE containers SET status = 0, username = 0 WHERE status = 1')
				updateScenarioStatus = ('UPDATE scenarios SET scenario_status = 2 WHERE scenario_id = %s')
				cursor.execute(updateQueueStatus)
				cursor.execute(updateResultStatus,(scena_id,))
				cursor.execute(updateContainerStatus)
				cursor.execute(updateScenarioStatus,(scena_id,))
				con.commit()
				
				# get info swarm untuk sshpass
				getConnection = ('SELECT swarm_ip, swarm_username, swarm_password FROM swarms WHERE is_used = 1')
				cursor.execute(getConnection)
				swarm_conn = cursor.fetchall()
				# run sshpass untuk copy gambar
				for swarm in swarm_conn:
					ip = swarm[0]
					user = swarm[1]
					passw = swarm[2]
					from_path='/home/cloudy/tugas-akhir-2019/implementasi/output/ss'+scena_id+'.png'
					to_path='/home/cloudy/web-tugas-akhir-2019/public/images/'
					command_line = 'sshpass -p %s scp %s@%s:%s %s' % (passw, user, ip, from_path, to_path)
					os.system(command_line)

def main():
	check = checkDb()
	if check > 0:
		getQueue()
		runScenario()
		calculateResult()
	else:
		print('Masih Kosong!')

if __name__ == '__main__':
	main()
\end{lstlisting}

\begin{lstlisting}[frame=single,tabsize=2,breaklines,caption={Isi berkas run\_scenario.py},label=runscenpy, captionpos=b, language=json]
from connection import connect
from sys import argv
import os

def main():
	# variabel argv
	username = argv[1]
	ip = argv[2]
	
	# connection
	con = connect()
	cursor = con.cursor()
	
	# get id skenario yang akan diuji
	query = "SELECT scenario_id, scenario_worker FROM scenarios WHERE username = %s AND scenario_status = 1"
	cursor.execute(query,(username,))
	scenarios = cursor.fetchall()
	
	# get kontainer
	getContainers = "SELECT node_ip, container_id FROM containers WHERE node_ip = %s AND status = 1"
	cursor.execute(getContainers,(ip,))
	containers = cursor.fetchall()
	
	for scenario in scenarios:
		scenid = scenario[0]
		worker = scenario[1]
		for container in containers:
			host = container[0]
			contid = container[1]
			command_line = 'docker container exec %s node getdata/index.js %s %s %s %s' % (contid, scenid, contid, worker, host)
			os.system(command_line)

if __name__ == '__main__':
	main()
\end{lstlisting}

\section*{Kode Script Penyimpan Data Kontainer} \label{scriptsh}

\begin{lstlisting}[frame=single,tabsize=2,breaklines,caption={Isi berkas save\_containers.sh},label=containersh, captionpos=b, language=json]
#!/bin/bash
set -e

SERVICE_NAME=$1;

TASK_ID=$(docker service ps --filter 'desired-state=running' $SERVICE_NAME -q)
NODE_ID=$(docker inspect --format '{{ .NodeID }}' $TASK_ID)
CONTAINER_ID=$(docker inspect --format '{{ .Status.ContainerStatus.ContainerID }}' $TASK_ID)
NODE_IP=$(docker inspect --format '{{ .Status.Addr }}' $NODE_ID)
NODE_HOST=$(docker inspect --format '{{ .Description.Hostname }}' $NODE_ID)

echo $TASK_ID | tr ' ' '\n' > 1.txt
echo $NODE_ID | tr ' ' '\n' > 2.txt
echo $CONTAINER_ID | tr ' ' '\n' > 3.txt
echo $NODE_IP | tr ' ' '\n' > 4.txt
echo $NODE_HOST | tr ' ' '\n' > 5.txt

sed 's/[^ ][^ ]*/"&"/g' 1.txt > taskid.txt
sed 's/[^ ][^ ]*/"&"/g' 2.txt > nodeid.txt
sed 's/[^ ][^ ]*/"&"/g' 3.txt > conid.txt
sed 's/[^ ][^ ]*/"&"/g' 4.txt > nodeip.txt
sed 's/[^ ][^ ]*/"&"/g' 5.txt > nodehost.txt

paste -d',' taskid.txt nodeid.txt conid.txt nodeip.txt nodehost.txt > data.txt

awk '{new="INSERT INTO containers (task_id, node_id, container_id, node_ip, node_host) VALUES ("$1");"; print new}' data.txt > data.sql

mysql -ucloudy -psembarang12 -h178.128.123.143 tugas_akhir_2019 -e "truncate table tugas_akhir_2019.containers"
mysql -ucloudy -psembarang12 -h178.128.123.143 tugas_akhir_2019 < data.sql

rm *.txt
rm data.sql
\end{lstlisting}

	\appendix

	\backmatter % Lampiran tanpa judul LAMPIRAN X, biasanya untuk BIODATA PENULIS
	\chapter{BIODATA PENULIS}
	\begin{wrapfigure}{l}{0.3\textwidth}
		\includegraphics[width=0.29\textwidth]{img/CPH.jpg}
	\end{wrapfigure}
	
	\textbf{Cahya Putra Hikmawan}, akbrab dipanggil Awan lahir pada tanggal 09 September 1996 di Krembung, Kabupaten Sidoarjo. Penulis merupakan seorang mahasiswa yang sedang menempuh studi di Departemen Informatika Institut Teknologi Sepuluh Nopember. Memiliki hobi antara lain membaca novel dan memancing. Selama menempuh pendidikan di kampus, penulis juga aktif dalam organisasi kemahasiswaan, antara lain Staff Departemen Pengembangan Profesi Mahasiswa Himpunan Mahasiswa Teknik Computer-Informatika pada tahun ke-2 dan ke-3. Pernah menjadi staff dan badan pengurus harian National Logic Competition Schematics tahun 2016 dan 2017. Selain itu penulis pernah menjadi asisten dosen dan praktikum di mata kuliah Sistem Operasi dan Jaringan Komputer.
\end{document}

\end{document} % YAY, WELCOME TO REAL WORD :)
